\documentclass[tablesignature]{scrartcl}
\usepackage[utf8]{inputenc}
\usepackage[T1]{fontenc}
\usepackage{fixltx2e}
\usepackage{graphicx}
\usepackage{longtable}
\usepackage{float}
\usepackage{wrapfig}
\usepackage{soul}
\usepackage{textcomp}
\usepackage{marvosym}
\usepackage{wasysym}
\usepackage{latexsym}
\usepackage{amssymb}
\usepackage{hyperref}
\tolerance=1000
\usepackage{booktabs}
\usepackage[scaled]{beraserif}
\usepackage[scaled]{berasans}
\usepackage[scaled]{beramono}
\usepackage[usenames,dvipsnames]{color}
\usepackage{fancyhdr}
\usepackage{subfig}
\usepackage{listings}
\lstnewenvironment{common-lispcode}
{\lstset{language={HTML},basicstyle={\ttfamily\footnotesize},frame=single,breaklines=true}}
{}
\usepackage{paralist}
\let\itemize\compactitem
\let\description\compactdesc
\let\enumerate\compactenum
\usepackage[letterpaper,includeheadfoot,top=12.5mm,bottom=25mm,left=19mm,right=19mm]{geometry}
\pagestyle{fancy}
\providecommand{\alert}[1]{\textbf{#1}}

\title{Setting Up Zambia from Scratch}
\author{Percy, Bendyogagirl}
\date{July 2011}

\begin{document}

\maketitle

% Org-mode is exporting headings to 3 levels.

\pagenumbering{roman}
\thispagestyle{fancy}
\renewcommand{\headrulewidth}{0pt}
\renewcommand{\footrulewidth}{1pt}
\lhead{}
\rhead{}
\chead{}
\lfoot{{Percy, Bendyogagirl} <{NELA.Percy@gmail.com}>}
\cfoot{}
\rfoot{\thepage}
\begin{abstract}
\vspace{5cm}
{\LARGE{\textbf{Abstract:\\}}}
Zambia is a piece of Con Management Software.  This document is a ``How To'' guide to help set up your Zambia FFF-branch instance from scratch for your convention.  This is still a work in progress.
\end{abstract}
\newpage
\renewcommand{\headrulewidth}{1pt}
\chead{{Setting Up Zambia from Scratch}}
\tableofcontents
\listoftables
\listoffigures
\newpage
\pagenumbering{arabic}
\section{Before beginning}
\label{sec-1}


  There are some required programs to run the Zambia software.  The
  following programs must be running on whatever server is going to
  serve up your Zambia instance: (LAMP or WAMP (WAMP is untested))
\begin{itemize}
\item apache
\item php
\item mysql
\item some form of email sending software (MTA)
\item SFTP/SCP
\end{itemize}

  Either on your staging machine (where you will be uploading the
  information from) or on your server (if you are loading directly to
  there) you must also have:
\begin{itemize}
\item svn
\item text editing program
\end{itemize}
\subsection{Definitions}
\label{sec-1_1}

\begin{longtable}{|p{3.5cm}|p{13.4cm}|}
\caption{Acronyms and Definitions} \label{tbl:acronymsdefinitions}\\
\hline
 Acronym or Term  &  Definition                                                                                                                                       \\
\hline
\endhead
\hline\multicolumn{2}{r}{Continued on next page}\
\endfoot
\endlastfoot
\hline
 precis           &  An element in your schedule.  Could be a class, panel, gathering, party, room-coverage, lounge, or whatever else you might have on your schedule  \\
 LAMP             &  Linux + Apache + MySQL + PHP/Perl/Python                                                                                                          \\
 WAMP             &  Windows + Apache + MySQL + PHP/Perl/Python                                                                                                        \\
\hline
\end{longtable}
\subsection{Decisions}
\label{sec-1_2}

   You will need to decide the following before you install initially:
\subsubsection{Standalone/Combination}
\label{sec-1_2_1}

    Is this system going to be a standalone system, or are you going
    to interface it with Congo, or another piece of registration software?
\subsubsection{Single/Multiple installation}
\label{sec-1_2_2}

    Is this going to be a single instantce of Zambia, or are you going
    to have multiple instances of Zambia running on the same machine
    with some shared resources?
\subsubsection{Webserver}
\label{sec-1_2_3}

    Where is your web installation location (whatever directory you
    will be putting your files into)?  Often something like
    /var/www/Zambia-Con or /home/username/public\_{}html/Zambia-Con or
    the like.
\subsubsection{db\_{}name.php file}
\label{sec-1_2_4}

    This information will go in the db\_{}name.php file when you install it.
\begin{itemize}

\item Required information for the program to run:
\label{sec-1_2_4_1}%
\begin{itemize}
\item Database hostname (\textbf{DBHOSTNAME}). If you are setting things up on
     the same machine that MySQL is running on, \emph{localhost} should be
     what you are using for \textbf{DBHOSTNAME}.
\item Database username (\textbf{DBUSERNAME}).  The user created for the
     Zambia program to access the database(s).
\item Database password (\textbf{DBPASSOWRD}).  The password created for the
     Zambia program to access the database(s).
\item Main database name (\textbf{DBDB}).  The database for all the single
     convention specific information to live in.
\item Biographical information database name (\textbf{BIODB}).  The database
     for all the (possibly shared between Zambia instances)
     biographical data.  If you are having the same people either for
     multiple years, or for multiple activities, it makes sense to
     centralize the information, otherwise it could be a one-shot for
     the particular convention.
\item The Reports databasse name (\textbf{REPORTDB}) points to where the
     reports are kept.  It is on of the main ways of interation with
     the data about the convention.  If you have multiple Zambia
     instances it makes sense for this to be a single, shared
     resource.
\item The CongoDump database name (\textbf{CONGODB}). Congo is another piece of
     software designed to interact with Zambia. The information in
     that set of databases can be accessed in a variety of ways.
\item The Limits (\textbf{LIMITDB}) and Localizations (\textbf{LOCALDB}) databases
     are used to key things that are specific to a convention, but
     might all want to live in a central database, for ease of
     deploying new conventions.
\item The TimeCard database (\textbf{TIMECARDDB}) might want to be specific to
     a convention, or generalized across many different activities of
     your organization, it often is local to the convention instance,
     but there are some reasons to have it in a single, externally
     served database.
\item The convention database keying tag (\textbf{CON\_{}KEY}) is useful when the
     Limits and Localizations are stored in a single table, to
     distinguish between which instance needs to be pulled.
\item your host name (\textbf{MYHOST}) (If you are setting things up on the same
     machine that MySQL is running on, \emph{localhost} should be what you
     are using for \textbf{MYHOST}) (this doesn't actually go in the file)
\end{itemize}


\item Extremely useful information:\\
\label{sec-1_2_4_2}%
Most of this information, currently in the db\_{}name.php file,
     would all be candidtates to be inthe Localization database.
\begin{itemize}
\item Con name.
\item Zambia administrator email.
\item Brainstorm email.
\item Programming email.
\item Registration email.
\item Number of days the con will run (code works for 1-8 currently.)
\item Date and time the con will start (In the format of yyyy-mm-dd
       HH:MM:DD for parsing purposes.  Suggested 00:00:00 for the
       start time.)
\item URL of the con (without the leading \href{http://}{http://} in the URL.)
\item Logo for the con (gif, png, etc.)
\item Availabilty Records (starting number of ``availability'' fields
       to render in the ``when I am available'' form, 8 is a good
       default.)
\item Are kids avaiable (This should probably be set to ``FALSE'', it
       is a hold-over from backwards compatibility.)
\item Default Duration of the classes. (How long, in H:MM format that
       the classes are expected to be.)
\item Duration in Minutes (Should probably be ``FALSE'': TRUE: in mmm;
       False: in hh:mm - affects session edit/create page only, not
       reports.)
\item Grid Spacer (The time divisions in the fixed grid produced, in
       seconds.  For example 1800 is 60 sec/min and 30 min, and a good
       default.)
\end{itemize}


\item Very useful information
\label{sec-1_2_4_3}%
\begin{itemize}
\item Guests of Honor badgelist (if you have specific featured Guests
       of Honor, the badgeids get listed here, comma seperated.)
       (This is being migrated as a flag for a presenter.)
\item Prefered total number of sessions upper limit (so your
       presenters don't oversubscribe themselves 5 is good default for
       a 3-4 day con.) (This should move to the Limits database.)
\item Prefered daily number of sessions upper limit (3 is a good
       default.) (This should move to the Limits database.)
\end{itemize}


\item Description minimums and maximums\\
\label{sec-1_2_4_4}%
All of this information should be migrated to the Limits
     database.  Only set a value if you need it, unset values are
     simply presume there is no limit to the information in that
     direction.  At some point, the precis descriptions will be folded
     into the same, or a similar structure to the Biographical
     Information is, currently.
\begin{itemize}
\item Minimum web biography character length (Some cons have minimum
       biographical information requirements.)
\item Maximum web biography character length (Some cons have a
       different limits for what is on the web, and what is in the
       book, 3000 characters is a good starting default for the WWW.)
\item Minimum book biography character length (Some cons have minimum
       biographical information requirements.)
\item Maximum book biography character length (if there isn't a
       difference in the limit, but still, there is a limit, set to
       the same as the web maximum.)
\item Minimum uri biography character length (the URI block is often
       unlimited in either direction.)
\item Maximum uri biography character length (the URI block is often
       unlimited in either direction.)
\item Minimum picture biography character length (the picture line is
       often unlimited in either direction.)
\item Maximum picture biography character length (the picture line is
       often unlimited in either direction.)
\item Minimum web precis character length (You need a precis
       description of at least this long to be acceptable, 10 as a
       good default.)
\item Maximum web precis character length (3000 as a good default.)
\item Minimum book precis character length (if there isn't a
       difference in the limit, set it to the same as above for these,
       as well.)
\item Maximum book precis character length
\item Minimum precis title character length (You need a precis title
       of at least this long to be acceptable, 5 as a good default.)
\item Maximum precis title character length (50 as a good default.)
\item Minimum name character length (if there is a need to make sure
       all names are of at least a specifc length.)
\item Maximum name character length (if there is a need to make sure
       all names are no more than a specifc length.)
\end{itemize}


\item Other interesting settings\\
\label{sec-1_2_4_5}%
The linguistic settings (below) will go away once the precis
     descriptions are migrated to a similar format as the Biographical
     information currently resides.
\begin{itemize}
\item Is this a bilingual event (This should probably be set to
       ``FALSE'' due to it's lack of complete support across the system,
       and the next few elements in this list, ignored.)
\item What the second language is.
\item Title caption in second language.
\item Description caption in second language.
\item Biography caption in second language.
\end{itemize}


\item The rest of the file should not have to change.\\
\label{sec-1_2_4_6}%
\end{itemize} % ends low level
\section{Downloading}
\label{sec-2}

  If you are checking the code out directly on the hosting machine,
  replace the final ``Zambia'' with what you decided the web install
  location will be.  If not, you can rename it to whatever you wish to
  call your staging area.

  Please, check out the code from:

  svn co \href{https://zambia.svn.sourceforge.net/svnroot/zambia/branches/FFF/}{https://zambia.svn.sourceforge.net/svnroot/zambia/branches/FFF/} Zambia
\section{Local file setup}
\label{sec-3}

  There are certain localisms you want to set up, outside the svn
  tree. This is so (should you need to) if an upgrade to the code-base
  is desired, it can be done, without writing over your
  customizations.

  To begin the process change to your web install location.  There you
  should see a list of files, and you will add one called ``Local''.
  When done your list of files should be:
  These files you will have to create or modify.
\subsection{db\_{}name.php}
\label{sec-3_1}

     Copy the example webpages/db\_{}name\_{}sample.php to Local/db\_{}name.php
     as a start, and then put in the values you decided upon before
     starting the install process.
\subsection{FooterTemplate.html}
\label{sec-3_2}

     This is where you will put your standard footer, that will be
     below all the public pages, to customize the look and feel to
     match your event's presentation.
\subsection{HeaderTemplate.html}
\label{sec-3_3}

     This is where you will put your standard header, that will be
     above all the public pages, to customize the look and feel to
     match your event's presentation.
\subsection{Participant\_{}Images}
\label{sec-3_4}

     If you choose to have images of your participants with their
     bios, make this directory, and any pictures that match the
     badgename of the participant will be put next to the bios.
\subsection{Verbiage}
\label{sec-3_5}

     If you wish to customize what is put forth to your participants,
     many of the pages allow for customization.  The list of them will
     grow as more are done.  Any information in these files will
     replace the default text.  Some examples of these files are:
\subsubsection{BrainstormWelcome\_{}0}
\label{sec-3_5_1}

\begin{verbatim}
<P> Here you can submit new suggestions or look at existing ideas for
panels, Meet and Greets, Special Interest Groups, Birds Of a Feathers,
Author Readings, and Author Signings.</P>
<P> As suggestions come in and we read through them, we will rework
them, combine similar ideas into a single item, split large ones into
pieces that will fit in their alloted time, etc.  Please expect the
suggestions you submit to evolve over time.</P>
<P> Also, please note that we always have more suggestions than are
physically possible with the space and time we have, so not everything
will make it.  We do save good ideas for future conventions.</P>
<UL> 
  <LI> <A HREF="BrainstormSearchSession.php">Search</A> for similar
  ideas or get inspiration.
  <LI> Email <A HREF="mailto:program@ourcon.org">
  program@ourcon.org</A> to suggest modifications on existing
  suggestion.
  <LI> <A HREF="BrainstormCreateSession.php">Enter a panel, MnG, SIG,
  BOF, et al suggestion.</A>
  <LI> <A HREF="BrainstormSuggestPresenter.php">Enter a suggestion for
  a Presenter.</A>
  <LI> See the list of <A HREF="BrainstormReportAll.php">All</A>
  suggestions (we've seen some and not see others).
  <LI> See the list of <A HREF="BrainstormReportUnseen.php">New</A>
  suggestions that have been entered recently (may not be fit for
  young eyes, we haven't see these yet).
  <LI> See the list of <A HREF="BrainstormReportReviewed.php">
  Reviewed</A> suggestions we are currently working through.
  <LI> See the list of <A HREF="BrainstormReportLikely.php">Likely to
  Occur</A> suggestions we are or will allow participants to sign up
  for.
  <LI> See the list of <A HREF="BrainstormReportScheduled.php">
  Scheduled</A> suggestions.  These are very likely to happen at con.
  <LI> Email <A HREF="mailto:vols@ourcon.org">vols@ourcon.org</A> to
  volunteer to help process these ideas.
</UL>
\end{verbatim}
\subsubsection{Introduction\_{}Blurb\_{}0}
\label{sec-3_5_2}

\begin{verbatim}
and before I introduce our speaker, let me ask, How many of you are
new?  Well, let me tell you, you are in for one heck of a ride.<br
\end{verbatim}
\subsubsection{Schedule\_{}Blurb\_{}0}
\label{sec-3_5_3}

\begin{verbatim}
Welcome to the Circus Fantastique.  We really appreciate all your
efforts to make this weekend go so well.  Below is your schedule for
the weekend.  If you have any questions, please, do not hesitate to
find our staff in the Green Room, or whomever our point-person is at
that time.  </P><P>I hope you will have all the fun you can!<hr>
\end{verbatim}
\subsubsection{StaffPage\_{}0}
\label{sec-3_5_4}

\begin{verbatim}
<P> Please note the tabs above.  One of them will take you to your
participant view.  Another will allow you to manage Sessions.  Note
that Sessions is the generic term we are using for all Events,
Classes, Panels, BOF/SIG/MnG, other activities, etc. </P>

<P>Current roles:
<UL>
  <LI>Pre-con Logistics: That tall guy, with the 'stash
  <LI>At-con Logistics: Bill(1)
  <LI>Speaker Liaison: Kat (with a "K")
  <LI>Assistant Speaker Liaison: Bill(2)
  <LI>Volunteer Captain: Cat
  <LI>Assistant Volunteer Captain: The Other Cat
  <LI>Green Room Czar: Tim
  <LI>Point People: Helium 1, Helium 2, and the Stupid But Cute.
  <LI>Schedule Wranglers: The group as a whole
  <LI>Technical support: Will
  <LI>(Tentative) Technical Support: Nyot
  <LI>Bio/copy editing: Rupert
</UL></P>

<P>The general flow of sessions over time is: <UL> <LI>Brainstorm -
New session idea put in to the system by one of our brainstorm
users. The idea may or may not be sane or good.  It could be too big
or too small or duplicative.

  <LI>Edit Me - New session idea that a participant or staff member
entered.  An idea entered by a brainstorm user that is non-offensive
should be moved to this status.  These are still rough and may well
have issues.  Still could be duplicates.

  <LI>Vetted - A real session that we would like to see happen.  At
this point the language should be fairly close to final in the
description. Spell checking and grammar checking should have happened.
It needs have publication status, a type, kid category, division and a
room set.  Please check the duration (defaults to 1 hour) and the
various things the session might need (like power, mirrors, etc.)
This is the minimal status that participants are allowed to sign up
for.  Avoid duplicates (however the list is still approximately 3
times what will actually run).

  <LI>Assigned - Session has participants assigned to it.

  <LI> Scheduled - Session is in the schedule (do not set this by hand
as the tool actually sets this for you when you schedule it in a
room!)  The language needs to match what you want to see
<b>published</b>.

</UL>
\end{verbatim}
\subsubsection{Volunteer\_{}Jobs\_{}0}
\label{sec-3_5_5}

\begin{verbatim}
<UL>
<H3>Introducer: (in room)</H3>
  <LI> Sign in at the Green Room, so we know everything is covered.
  <LI> Collect anything needful, like handouts and blank surveys, or
  if it is the first class of the day, the signs, from the Green Room.
  <LI> Be at class 10 minutes early (at the actual end of the previous
  class).
  <LI> You may, if you wish, pre-stage surveys on people's seats for
  when they arrive.
  <LI> At the beginning of class, move to the front of the room and do
  the introduction.  The Con Blurb and the speaker(s) bio(s) as
  provided.
  <UL>
    <LI>NOTE: A board member or member of the organizing team may step
    forward to do the introduction, in which case, please hand them
    the paper to do it off of.
  </UL>
  <LI> Take the head count of the class (twice) and write them in the
  spots provided on the introduction paper.
  <LI> Be in the back of the room during class so:
  <UL>
     <LI> When the Runner comes to check on the room, you can let them
     know if there is anything needed.
     <LI> If there is vending in the class, you might need to mind the
     table, while the presenter is presenting, if they don't already
     have someone assisting them.
     <LI> Using the signs provided, give the 10 minute, 5 minute, and
     Done warnings.
     <LI> Hand out/collect surveys and pencils at the end of class.
  </UL>
  <LI> Do the hand-off to the next Introducer, which includes the
  blank surveys and the signs.
  <LI> Return the filled out surveys collected folded in the class
  sheet, when you are checking in at the end of your stint.
<H3>Volunteer: (outside room)</H3>
  <LI> Check that people coming into the room have the correct
  wristbands.  If they do not, politely send them to registration (if
  it is open) to get them.  If there is an issue, notify the point
  person.
  <LI> Stay at the door during class to ensure that excessive ins and
  outs don't occur.
<H3>Runner: (all over con)</H3>
  <LI> Ensure that every class room has what it needs.
  <LI> Ensure that that A/V and supply needs of a class are met prior
  to it beginning.
  <LI> You can quietly and respectfully bring any supplies into the
  room as a class is going on, and make sure the Introducer knows what
  was delivered.
<H3>Green Room: (green room)</H3>
The green room is a space designated for Presenters, Programming
Volunteers, Panelists, and Assistants only.  While you are welcome to
hang out there, it is also the Programming Team's Ground Zero, so, you
might be pressed into service.
  <LI> Assist the Program Participants, including disseminating their
  packets as necessary.
  <LI> Check in and out the Introducers and Volunteers as they come on
  and off their stints.
  <LI> Make sure all necessary supplies are available for the
  volunteers as they arrive for check in, including any handouts.
  <LI> Be available to collect the surveys, etc as they arrive.
  <LI> Stay in the room and hang out with everyone!
  <LI> Be in contact with the Programming Point Person for any
  problem.
</UL>
\end{verbatim}
\subsubsection{Welcome\_{}Letter\_{}Presenters\_{}0}
\label{sec-3_5_6}

\begin{verbatim}
<P>Dude!

<P>Thanks for helping us, man.  You really came through.  Like,
everyone learned bags of info, and your flow was rad!

<P>Every hand was good, yours were great!  High-five!

<P>Dude!

<P>The org-folk.
\end{verbatim}
\subsubsection{Welcome\_{}Letter\_{}Presenters\_{}and\_{}Volunteers\_{}0}
\label{sec-3_5_7}

\begin{verbatim}
<P>We would like to express our gratitude for your contribution to the
Ancient Order of the Spies Convention.  Your expertise, wisdom,
experience, and willingness to share your knowledge are critical
elements in what will make our event a success.  We here at Opsidec do
all we can to create a safe and inviting environment for all
secretive/spying/hiding people, but we must also rely on support from
generous allies such as yourself.  Your time and effort are much
appreciated, and are a benefit to all in this lifestyle.

<P>You contribution is helping us to create an event in which any and
all people can learn and access information that they may not have
available to them in their general life.  This process is crucial to
expand knowledge and support throughout our cities and also throughout
the world.  You are assisting in constructing a safe and supportive
atmosphere that truly fosters our community.  Thank you again for your
participation in the Ancient Order of the Spies Convention.  Your
addition to this event is an advantage to all.

<P>With much gratitude,

<P>Opsidec Limited Organizers
\end{verbatim}
\subsubsection{Welcome\_{}Letter\_{}Volunteers\_{}0}
\label{sec-3_5_8}

\begin{verbatim}
<P>We would like to express our gratitude for your contribution to the
Con of your Dreams.  Your willingness to share your time and energy
are critical elements in what will make our event a success.  We here
at Dream Productions do all we can to create a safe and inviting
environment for all sleapers, but we must also rely on support from
generous allies such as yourself.  Your effort is much appreciated,
and is a benefit to all in this lifestyle.

<P>You contribution is helping us to create an event in which any and
all people can learn and access information that they may not have
available to them in their general life.  This process is crucial to
expand knowledge and support throughout our cities and also throughout
the world.  You are assisting in constructing a safe and supportive
atmosphere that truly fosters our community.  Thank you again for your
participation in the Con of your Dreams.  Your addition to this event
is an advantage to all.

<P>With much gratitude,

<P>The Programming Team
\end{verbatim}
\section{Database setup}
\label{sec-4}

  You should already have mysql set up.  If mysql is not already set
  up, a good guide to setting up a mysql server is:

\begin{small}
  \href{http://www.linuxhomenetworking.com/wiki/index.php/Quick_HOWTO_:_Ch34_:_Basic_MySQL_Configuration}{http://www.linuxhomenetworking.com/wiki/index.php/Quick\_HOWTO\_:\_Ch34\_:\_Basic\_MySQL\_Configuration}
\end{small}

  The pieces of information you will need are from the above decisions
  for the db\_{}name.php file:

\begin{itemize}
\item database hostname: (\textbf{DBHOSTNAME}) (If you are setting things up on
    the same machine that MySQL is running on, \emph{localhost} should be
    what you are using for \textbf{DBHOSTNAME}).
\item database username: (\textbf{DBUSERNAME})
\item database password: (\textbf{DBPASSOWRD})
\item database name: (\textbf{DBNAME})
\item Each of the alternate database locators that will be used:
    (\textbf{BIODB}), (\textbf{REPORTDB}), (\textbf{CONGODB}), (\textbf{LIMITDB}), (\textbf{LOCALDB}),
    (\textbf{TIMECARDDB})
\item your host name (\textbf{MYHOST}) (If you are setting things up on the
    same machine that MySQL is running on, \emph{localhost} should be what
    you are using for \textbf{MYHOST})
\end{itemize}
\subsection{Hosted server}
\label{sec-4_1}

  If you are going to have your database served from a machine that
  is running cpanel or some other menu-based software, the method of
  setting up your database should be documented there.

  The chances are your setup will have you:
\begin{itemize}
\item create a database or databases
\item possibly create a MySQL user
\item add MySQL user to the database or databases
\item grant the MySQL user all privs.
\end{itemize}
\subsection{Your own MySQL Setup}
\label{sec-4_2}

   If you are setting up your own MySQL server, and need to set up the
   database by hand the following steps should work for you.  Don't
   forget to replace the instances of \textbf{DBHOSTNAME}, \textbf{DBUSERNAME},
   etc. with the proper bits of information.

\begin{itemize}
\item Log into the database: (it should ask you for your MySQL root password)
\end{itemize}

\begin{verbatim}
mysql -h*DBHOSTNAME* -p -u root
\end{verbatim}


\begin{itemize}
\item Create your database:
\end{itemize}

\begin{verbatim}
create database DBNAME;
\end{verbatim}


\begin{itemize}
\item Grant \textbf{DBUSERNAME} user access with the password of \textbf{DBPASSWORD}:
\end{itemize}

\begin{verbatim}
grant all on DBNAME.* to 'DBUSERNAME'@'MYHOST' identified by 'DBPASSOWRD';
grant lock tables on DBNAME.* to 'DBUSERNAME'@'MYHOST';
\end{verbatim}


\begin{itemize}
\item Reset the privilages
\end{itemize}

\begin{verbatim}
flush privileges;
\end{verbatim}
\section{Database populate}
\label{sec-5}

  change directories until you are in the Install directory, then:
\begin{verbatim}
mysql -hDBHOSTNAME -p -uDBUSERNAME DBNAME < ./EmptyDbase.dump
\end{verbatim}
\section{Database tweaks}
\label{sec-6}

  Some of the tables in the database don't yet have appropriate
  front-ends, so, to customize them for your particular event, you
  will need to modify them directly from the MySQL client.  As
  development proceeds, these will get fewer over time.

  Currently, they are:
\begin{itemize}
\item Divisions:: If you want some other divisions than Other,
    Programming, Events, Fixed Functions, Hotel, Unspecified, and
    Volunteer.
\item EmailCC:: Needs to be customized for your convention.
\item EmailFrom:: Needs to be customized for your convention.
\item EmailTo:: Might need to be customized.
\item Features:: List of things that can be in a room.  Might need to be
    customized.
\item Phases:: The ``Phase'' you are in will need to be changed as your
    phase changes.
\item PreconHours:: If you are tracking volunteer hours, the PreconHours
    will probably need to be added to.
\item PubStatuses:: Depending on the useage of the software, you might
    need more statuses than Prog Staff, Public, Do Not Print, and
    Volunteer.
\item QuestionsForSurvey:: You might want to change these.
\item RegTypes:: Depending on how you use it, the RegTypes may change.
\item Roles:: Fairly standard, but might want to be customized for your
    convention.
\item RoomSets: Fairly standard, but might want to be customized for
    your convention.
\item Rooms:: This definitely wants to be customized for your
    convention.
\item Services:: List of services that can be provided to a room.  Might
    need to be customized.
\item SessionStatuses:: Might need to be customized for your
    convention.
\item Tracks:: Probably will want to be customized for your convention
\item Types:: May want to be customized for your convention.
\end{itemize}

  Also, some of the Permission interconnects might have to be
  customized for your convention.

  One set of tables that you might be updating across the life of this
  instance of Zambia is the Reports table.  As people generate useful
  reports, they do tend to get shared.  We hope that, should you
  develop noteworthy reports, you share them back with the community
  at large, as well.

  Loading such reports are often as simple as:
\begin{verbatim}
mysql -hDBHOSTNAME -p -uDBUSERNAME DBNAME < ./NewReports.sql
\end{verbatim}



  Sharing them is as simple as, say, exporting your new report called
  \emph{voltimepanelists}:
\begin{verbatim}
echo "SELECT * FROM Reports WHERE reportname='voltimepanelists';" | 
mysql -hDBHOSTNAME -p -uDBUSERNAME DBNAME > ./NewReports.sql
\end{verbatim}
\section{Account creation}
\label{sec-7}
\subsection{Standalone}
\label{sec-7_1}

   If you are going to be using Zambia and not some other registration
   package, you are going to need access to the program, to begin
   adding the people who are going to be working with the system.

   Currently the easiest way to do so is to add the first three users,
   by pulling in the Initial\_{}Users.sql file from the \emph{Install}
   directory.

\begin{verbatim}
mysql -hDBHOSTNAME -p -uDBUSERNAME DBNAME < ./Inital_Users.sql
\end{verbatim}



   Once you have done that, you can log in to Zambia using the badgeid
   of \textbf{101} and the password of \textbf{changeme}.

   You then can modify the appropriate information.  Under the \emph{Manage    Participants \& Schedule} tab, there is an \emph{Administer Participants}
   choice.  Selecting that will allow you to update your password
   (\underline{important step}) and the ``Edit Further'' link at the bottom of
   the page will allow you to update the information so it actually
   matches you.

   Feel free to then go and add the rest of your staff, off of the
   \emph{Enter Participants} link.
\subsection{Congo}
\label{sec-7_2}

   You might want to complete the activites above, just to make sure
   you have access, but once you do, you can migrate the congo data
   into the system, so all the other folks have their information
   added.

   From congo, do:
\begin{verbatim}
export_program_participants_congo.sql
\end{verbatim}



   This generates sql that can be, in turn, locaded into Zambia.
\subsection{Not Congo}
\label{sec-7_3}

   Tying this into another registration system is slightly more
   complicated.  The easiest way is to use the ``regtype'' field to
   track the registration number that the various other registration
   programs give you, and see if there is a way to massage their data
   into the ``CongoDump'' format.
\section{First steps}
\label{sec-8}
\subsection{Schedule}
\label{sec-8_1}

   Establishing the schedule of activities in the form of a ``todo''
   list is probably the first thing you wish to do.

\begin{verbatim}
YourWebPath/webpages/genreport.php?reportname=tasklistdisplay
\end{verbatim}



   Replacing, of course \emph{YourWebPath} with the proper URL to get to
   your Zambia-FFF branch install.
\subsection{Brainstorming}
\label{sec-8_2}

   The Brainstorming links should work immediately.  From the top
   directory (index) page of your site, you should be able to click on
   the ``Suggest a Session/Presenter'' button and get right into it.
\section{Backing up}
\label{sec-9}

  Under the \emph{scripts} directory there is a nice little shell-script
  that you can call with cron to back your information up.  If you are
  to use it, make sure you create the \emph{Data\_{}Backup} directory under
  the \emph{Local} directory before you use it.  I back up weekly several
  months before the con, start in on daily once heavy changes are
  being made, so we loose less information if there is a problem, and
  then about a month or so after the con, back off to weekly or
  monthly.  At one point in time, I was running it hourly, just to be
  sure.

  The script is invoked as:
\begin{verbatim}
backup_mysql /your/path/to/Zambia-FFF/instance
\end{verbatim}
\section{Quick and dirty}
\label{sec-10}

  To build a quick and dirty copy on the same machine as another running version: 
\begin{itemize}
\item for i in ../Zambia-FFF/* ; do ln -s \$i . ; done
\item rm Local
\item rm webpages
\item mkdir Local webpages
\item cd webpages
\item for i in ../../Zambia-FFF/webpages/* ; do ln -s \$i . ; done
\item cd ../Local
\item mkdir Verbiage
\item for i in ../../Zambia-FFF/Local/*.html ; do ln -s \$i . ; done
\item ln -s ../../Zambia-FFF/Local/Participant\_{}Images .
\end{itemize}

  Finally, copy over, and modify the db\_{}name.php as appropraite
  To update them all do:
\begin{verbatim}
for i in FFF-[34]*
  do 
    pushd $i/webpages
    for j in ../../Zambia-FFF/webpages/*
      do 
        ln -s $j .
      done
    popd
  done
\end{verbatim}

\end{document}
