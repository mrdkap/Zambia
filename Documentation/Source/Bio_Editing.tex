\documentclass[tablesignature]{scrartcl}
\usepackage[utf8]{inputenc}
\usepackage[T1]{fontenc}
\usepackage{fixltx2e}
\usepackage{graphicx}
\usepackage{longtable}
\usepackage{float}
\usepackage{wrapfig}
\usepackage{soul}
\usepackage{textcomp}
\usepackage{marvosym}
\usepackage{wasysym}
\usepackage{latexsym}
\usepackage{amssymb}
\usepackage{hyperref}
\tolerance=1000
\usepackage{booktabs}
\usepackage[scaled]{beraserif}
\usepackage[scaled]{berasans}
\usepackage[scaled]{beramono}
\usepackage[usenames,dvipsnames]{color}
\usepackage{fancyhdr}
\usepackage{subfig}
\usepackage{listings}
\lstnewenvironment{common-lispcode}
{\lstset{language={HTML},basicstyle={\ttfamily\footnotesize},frame=single,breaklines=true}}
{}
\usepackage{paralist}
\let\itemize\compactitem
\let\description\compactdesc
\let\enumerate\compactenum
\usepackage[letterpaper,includeheadfoot,top=12.5mm,bottom=25mm,left=19mm,right=19mm]{geometry}
\pagestyle{fancy}
\providecommand{\alert}[1]{\textbf{#1}}

\title{Bios Editing in Zambia}
\author{Percy}
\date{November 2011}

\begin{document}

\maketitle

% Org-mode is exporting headings to 3 levels.

\pagenumbering{roman}
\thispagestyle{fancy}
\renewcommand{\headrulewidth}{0pt}
\renewcommand{\footrulewidth}{1pt}
\lhead{}
\rhead{}
\chead{}
\lfoot{{Percy} <{nela.program@gmail.com}>}
\cfoot{}
\rfoot{\thepage}
\begin{abstract}
\vspace{5cm}
{\LARGE{\textbf{Abstract:\\}}}
Zambia is a piece of Con Management Software.  This document is a ``How To'' guide assisting in the way of editing the Bios for the Zambia FFF-branch instance for your convention.  This is still a work in progress.
\end{abstract}
\newpage
\renewcommand{\headrulewidth}{1pt}
\chead{{Bios Editing in Zambia}}
\tableofcontents
\listoftables
\listoffigures
\newpage
\pagenumbering{arabic}
\section{Introduction}
\label{sec-1}


  The biographical information for your presenters is an important
  part of the convention experience that Zambia manages.  This is part
  of how people decide they want to attend your convention and part of
  what they might want to do when there.  From a presenter's point of
  view, the biographical information is a way to connect with their
  fan-base, and allow contact, give information, or arrange for
  continued interactions.  From a con manager's point of view,
  managing biographical information is fairly work-intensive, and one
  of the things that doesn't exactly scale.

  There are several pieces that make up the biographical information
  matrix within Zambia.  Few, any, or all of the elements can be
  chosen to be deployed and customized to the desire of your
  particular convention or set of conventions.  Currently the
  biographical information information is designed to be held in a
  shared database, outside of a specific convention, so that, while
  the information is available and editable within that convention,
  when the information is updated, it is updated across all the
  conventions served from the same server, and information is
  available across multiple convention instances.  Not all of the
  conventions will necessarily use all of the information.
\section{Biographical Information Matrix}
\label{sec-2}


  The biographical information Matrix is made up of three different
  values.  The types of biographical information, the status of a
  biographical element, and the language that that element is written
  in.
\subsection{Types}
\label{sec-2_1}


   The (extendable) types that are currently in use in Zambia are:
\begin{itemize}
\item Web: The verbose biographical information, limited by variables
     in the db\_{}name.php file, that is shared on your Zambia site, for
     all to browse to, often less limited than the published
     information.  Most HTML markup works with this type.
\item Book: The biographical information that is published in your
     convention book, handout, or other printed materials that is
     often more limited by space concerns, and publishing costs.
     These limits are set in the db\_{}name.php file.  This will be
     published so markup and hypertext is discouraged.
\item URI: The set of Uniform Resource Identifiers (URLs, URNs, etc)
     that are available on your Zambia site, for people to follow.
     Most HTML markup works with this type.
\item Picture: This is a URI reference to either an outside (http)
     picture or, preferably, a copy of said picture residing on the
     serving machine (/path/to/picture), so if the original gets
     hacked, Zambia isn't compromised.
\end{itemize}
\subsection{States}
\label{sec-2_2}

\begin{itemize}
\item raw: This is the information provided by the presenter when they
     are permitted to, within Zambia.  This might be as far as the
     involvement with the biographical information that the Zambia
     staff will have.  This allows for unknown people to put unknown
     references and information on your web-site.  Beware.
\item edited: This is intended to be the mid-step, the negotiation
     point between the people responsible for what is being published
     in the literature and on the web, and the individual presenters,
     it is presented to them, so dialogue can happen, and they can see
     how it is to be presented, should it be approved.  This might be
     as far as the involvement with the biographical information that
     the Zambia staff will have.  Once in the edited mode it might be
     considered published.
\item good: This is intended to be what is published.  Once the edited
     matches the raw, it should be promoted to the good level, then if
     changes to the raw show up, they can go back through the vetting
     process and, once again, be promoted to good, and then available
     for the various publishing media.
\end{itemize}
\subsection{Languages}
\label{sec-2_3}


   The language field is designed to contain any of the various
   language elements.  Currently en-us and fr-ca are the two expected
   ones, as a hold-over from the original ``secondary language'' concept
   of Zambia when it was used for a Canadian event.  Starting any
   language will allow it to be flexible enough to have it show up in
   the appropriate tables.  Just adding to the LANGUAGE\_{}LIST searches
   for each/all of the languages listed.  While this is a first step
   in the internationalization of this software, expanding the rest,
   along these lines is expected and planned for.  Each of the other
   elements, the various verbiage on the pages, and in the reports
   will pull from similar tables, able to be customized off of, or
   feed concurrently, depending on the design the multiple languages
   available.  All of the biographical information is designed to be
   subsequently displayed, rather than switched on the
   \$\_{}SESSION['language'] variable.
\section{Pages}
\label{sec-3}
\subsection{StaffManageBios.php}
\label{sec-3_1}

\label{StaffManageBios.php}

   The starting point for managing the biographical information data.
   This is set to work with only those folks who will be published.
   This is all the presenters, and the any of the super-volunteers.
   They might not be chosen to be published, or they might be,
   depending on the choice of the convention, and the relative space
   in the various publications.

   The rows are organized around the states so that if there is
   missing raw, edited (if used), and good (if used) elements, they
   are reported first, then if the raw elements and the edited
   elements don't match (if used) or the edited elements and the good
   elements don't match (if used) they are reported next.  Then,
   lastly, if everything matches, that is reported last.  Hopefully,
   come publication time, and con-time, all the elements will be in
   the last row.

   The columns are a combination of the various languages available
   (somewhat useless if only one language is used, but there, as a
   place-holder) and the type of the element.

   To bring up the list of individuals in any particular category of
   missing, incorrect, or even correct information, simply click on
   the number of elements in the state you wish to work with.  In the
   example below, there are seven individuals who do not have an
   edited en-us web entry in the biographical information matrix.  By
   clicking on the ``7'' in that section of the table, you bring up the
   list of participants who's elements need editing.

   The table might resemble the following:
\begin{longtable}{|c|c|c|c|c|c|c|c|c|}
\caption{Staff - Manage Participant Biographies} \label{tbl:staffmanageparticipantbiographies}\\
\hline
 Count of the States of the bios    &  en-us web  &  fr-ca web  &  en-us book  &  fr-ca book  &  en-us uri  &  fr-ca uri  &  en-us picture  &  fr-ca picture \\
\hline
\endhead
\hline\multicolumn{9}{r}{Continued on next page}\
\endfoot
\endlastfoot
\hline
 Missing raw bio                    &             &             &           3  &           6  &         59  &         61  &             14  &             61  \\
 Missing edited bio                 &          7  &         16  &          21  &           5  &         48  &         61  &             14  &             61  \\
 Missing good bio                   &         59  &         59  &          59  &          59  &         59  &         61  &             14  &             61  \\
 Raw bio doesn't match edited bio   &         17  &         11  &          18  &          19  &          9  &             &                 &                 \\
 Edited bio doesn't match good bio  &         12  &          9  &          28  &          46  &          9  &             &                 &                 \\
 All bios match                     &          5  &          3  &           4  &              &             &         61  &             47  &             61  \\
\hline
\end{longtable}


   Once the particular subsection of the editing has been picked, you
   will be provided with a matrix of names in this category, so you
   might choose the one(s) to be edited.

   The three columns in this table are:
\begin{itemize}
\item Participant: If the link that is this name is chosen from this
     column, you will be brought to the \hyperref[StaffEditBios.php]{StaffEditBios.php} page
     relevant to the participant chosen, for editing purposes.
\item Edit Full: If the link that is this name is chosen from this
     column, you will be brought to the \hyperref[StaffEditCreateParticipant.php]{StaffEditCreateParticipant.php}
     page, in the mode of editing the participant chosen, in case that
     is necessary to find out more about who they are, or if there are
     any notes, or the like to assist you with the biographical
     information editing.
\item Currently being edited by: This shows you the individual who
     presumes that they are the one editing the particular
     information.  Please, do not choose to edit biographical
     information for someone who is locked by someone other than you.
\end{itemize}

   The table might resemble the following:
\begin{longtable}{|l|l|l|l|}
\caption{Staff - Manage Participant Biographies Subedit} \label{tbl:staffmanageparticipantbiographiessubedit}\\
\hline
 Participant  &  Edit Full    &  Currently being edited by \\
\hline
\endhead
\hline\multicolumn{3}{r}{Continued on next page}\
\endfoot
\endlastfoot
\hline
 Mr E.        &  Mr E.        &                             \\
 Joker        &  Joker        &  Str8mn                     \\
 Batman       &  Batman       &  The Riddler                \\
 The Riddler  &  The Riddler  &  Batman                     \\
 Catwoman     &  Catwoman     &                             \\
 Cartman      &  Cartman      &                             \\
 Robin        &  Robin        &                             \\
\hline
\end{longtable}


   There is also the ``return'' link, just above the table, that brings
   you back to the \hyperref[tbl:staffmanageparticipantbiographies]{Staff - Manage Participant Biographies}, in case you
   are done with this particular subset of individuals, and wish to
   deal with another.
\subsection{StaffEditBios.php}
\label{sec-3_2}

\label{StaffEditBios.php}

   If the edited and/or good states are to be used, this page is where
   an individual's information is edited.  If just the raw information
   is used, there is no need to use this page.  Once this page is
   opened for a participant, the editing individual's name will be
   placed in the ``Currently being edited by'' column, so two people
   don't try to edit the same bio information at the same time.  

   There are two links at the top of the page.  The first, being the
   individuals name, will lead you to the
   \hyperref[StaffEditCreateParticipant.php]{StaffEditCreateParticipant.php} page, in the mode of editing the
   participant chosen, and the return link, which will bring you right
   back to the table \hyperref[tbl:staffmanageparticipantbiographiessubedit]{Staff - Manage Participant Biographies Subedit} on
   the \hyperref[StaffManageBios.php]{StaffManageBios.php} page.

   Of course, any of the ``Save Whole Page'' buttons will save the
   current state of the entire page, and clear the ``Currently being
   edited by'' flag.

   This design of this page has a series of elements, broken down
   first by state element (raw, edited, good) state, then by language
   (en-us, fr-ca), and then by the type (web, book, uri, picture)
   allowing for the editing of the edited (if used) and good (if used)
   information.  The raw information is there for reference and
   copying purposes only and is not in a malleable state.
\subsection{StaffEditCreateParticipant.php}
\label{sec-3_3}

\label{StaffEditCreateParticipant.php}

   While this page is of more general use (which is documented
   elsewhere) if absolutely necessary, it can be used to modify the
   raw biographical information. This is most useful should that be
   the only state of the information chosen for this instance of
   Zambia.  The raw state of all of the types (web, book, URI,
   picture) and languages are available to be edited on this page.

\end{document}
