\documentclass[captions=tablesignature]{scrartcl}
\usepackage[utf8]{inputenc}
\usepackage[T1]{fontenc}
\usepackage{fixltx2e}
\usepackage{graphicx}
\usepackage{longtable}
\usepackage{float}
\usepackage{wrapfig}
\usepackage{rotating}
\usepackage[normalem]{ulem}
\usepackage{amsmath}
\usepackage{textcomp}
\usepackage{marvosym}
\usepackage{wasysym}
\usepackage{amssymb}
\usepackage{hyperref}
\tolerance=1000
\usepackage{booktabs}
\usepackage[scaled]{beraserif}
\usepackage[scaled]{berasans}
\usepackage[scaled]{beramono}
\usepackage[usenames,dvipsnames]{color}
\usepackage{fancyhdr}
\usepackage{subfig}
\usepackage{listings}
\lstnewenvironment{common-lispcode}
{\lstset{language={HTML},basicstyle={\ttfamily\footnotesize},frame=single,breaklines=true}}
{}
\usepackage{paralist}
\let\itemize\compactitem
\let\description\compactdesc
\let\enumerate\compactenum
\usepackage[letterpaper,includeheadfoot,top=12.5mm,bottom=25mm,left=19mm,right=19mm]{geometry}
\pagestyle{fancy}
\setcounter{secnumdepth}{3}
\author{Percy\thanks{NELA.Percy@gmail.com}}
\date{November 2011}
\title{Bios Editing in Zambia}
\hypersetup{
  pdfkeywords={Zambia, Documentation, FFF branch},
  pdfsubject={Zambia is a piece of Conference Management Software.  This document is a "How To" guide to editing the various bios type entries for the Zambia FFF-branch instance for your conference.  This is still a work in progress.},
  pdfcreator={}}
\begin{document}

\maketitle
\pagenumbering{roman}
\thispagestyle{fancy}
\renewcommand{\headrulewidth}{0pt}
\renewcommand{\footrulewidth}{0pt}
\lhead{}
\rhead{}
\chead{}
\lfoot{}
\cfoot{}
\rfoot{}
\begin{abstract}
\vspace{5cm}
{\LARGE{\textbf{Abstract:\\}}}
Zambia is a piece of Conference Management Software.  This document is a "How To" guide to editing the various bios type entries for the Zambia FFF-branch instance for your conference.  This is still a work in progress.
\end{abstract}
\newpage
\renewcommand{\headrulewidth}{1pt}
\renewcommand{\footrulewidth}{1pt}
\chead{
Bios Editing in Zambia
}
\lfoot{
Percy <NELA.Percy@gmail.com>
}
\rfoot{\thepage}
\setcounter{tocdepth}{3}
\tableofcontents
\listoftables
\newpage
\pagenumbering{arabic}
\section{Introduction}
\label{sec-1}

The biographical information for your presenters is an important
part of the convention experience that Zambia manages.  This is part
of how people decide they want to attend your convention and part of
what they might want to do when there.  From a presenter's point of
view, the biographical information is a way to connect with their
fan-base, to allow contact, give information, or arrange for
continued interactions.  From a con manager's point of view,
managing biographical information is fairly work-intensive, and one
of the things that doesn't exactly scale.

There are several pieces that make up the biographical information
matrix within Zambia.  Few, any, or all of the elements can be
chosen to be deployed and customized to the desire of your
particular convention or set of conventions.  Currently the
biographical information is designed to be held in a shared
database, which lies outside of a specific convention, so that,
while the information is available and editable within that
convention, when the information is updated, it is updated across
all the conventions served from the same server.  Therefore the
updated information is available across multiple convention
instances.  Not all of the conventions will necessarily use all of
the information.

\section{Biographical Information Matrix}
\label{sec-2}

The biographical information Matrix is made up of four different
values.  They are the types of biographical information, the status
of a biographical element, the destination for the biographical
element, and the language that that element is written in.  There
are limits applied to the various sizes of the entries, governed by
the PublicationsLimits table.
\subsection{Types}
\label{sec-2-1}

The (extendable) types that are currently in use are:
\begin{itemize}
\item uri: The set of Uniform Resource Identifiers (URLs, URNs, etc)
that are available on your Zambia site, for people to follow.
Most HTML markup works with this type.  Examples include mailto
references and webiste references.
\item bio: The written biographical descriptive information.
\item name: The name to be used in each circumstance
\item pronoun: The preferred pronoun.
\item twitter: A series of white-space separated twitter addresses.
\item facebook: A series of white-space separated Facebook addresses.
\item fetlife: A series of white-space separated FetLife addresses.
\end{itemize}

\subsection{States}
\label{sec-2-2}

The states that are currently in use are used as a flow for the
information and communication around such.
\begin{itemize}
\item raw: This is the information provided by or updated by the
presenter when they have been granted access to Zambia.  They
might update this, and that will show as differences between this
state and the "edited" state and/or the "good" state.  This
allows for unknown people to put unknown references and
information on your web-site.  Beware.
\item edited: This is intended to be the mid-step, the negotiation
point between the people responsible for what is being published
in the literature and on the web, and the individual presenters,
it is presented to facilitate dialogue, and everyone can see how
it is to be presented, should it be approved.  Once the "edited"
state and the "raw" state are in agreement due to the
collaboration between the staff and the participant, it would be
considered ready for promotion to the "good" state.
\item good: This is what is slated to be published.  Once the "edited"
state information matches the "raw" state information, that bio
element should be promoted to the "good" state. If further
changes to the "raw" state information happen, because a
participant has edited the information, the process begins again.
various publishing media.
\end{itemize}

\subsection{Destinations}
\label{sec-2-3}

The possible destinations for each of the types, determines where
that particular information is going to be used.  Each of the
limits set in the PublicationLimits table is both Type and
Destination keyed, so, for example, the limit for the information
published on the web might be more verbose (or in a different
format) than that published on the badge, or in the con hard copy
publications.
\begin{itemize}
\item web: To be published on the web, on the larger website, and in
the KonOpas subset, if that is chosen to be used.
\item book: To be published in a hard copy format, a book, a pamphlet,
or the like about the con.
\item badge: To be published on a name-badge type object, so that a
subset of the information is immediately available at a glance.
\item staffweb: As for the web, but for staff bios, which might be
different from present or vending bios.
\item staffbook: Similar to staffweb, for the staff bios which might be
different from presenting or vending bios.
\end{itemize}

\subsection{Languages}
\label{sec-2-4}

The language field is designed to contain any of the various
language elements.  Currently en-us and fr-ca are the two expected
ones, as a hold-over from the original "secondary language" concept
of Zambia when it was used for a Canadian event.  Starting any
language will allow it to be flexible enough to have it show up in
the appropriate tables.  Just adding to the LANGUAGE\_LIST searches
for each/all of the languages listed.  While this is a first step
in the internationalization of this software, expanding the rest,
along these lines is expected and planned for.  Each of the other
elements, the various verbiage on the pages, and in the reports
will pull from similar tables, able to be customized off of, or
feed concurrently, depending on the design the multiple languages
available.  All of the biographical information is designed to be
subsequently displayed, rather than switched on the
\$\_SESSION['language'] variable.

\section{Pages}
\label{sec-3}
\subsection{StaffManageBios.php}
\label{sec-3-1}
\label{StaffManageBios.php}
\underline{
\href{../webpages/StaffManageBios.php}{StaffManageBios.php}
}

This is the starting point for managing the biographical and
descriptive information data that will be published.  The specific
grids are:
\begin{itemize}
\item the presenters
\item the staff
\item the schedule elements
\item the volunteers
\item the vendors
\end{itemize}

Each of the tables on this page (illustrated below) address
different sets of information.  There might be some overlap, for
example a staff member might also be a vendor, or a presenter, so
information might be accessible via several paths.  Every table
cell will allow you to click through, and address whatever
biographical elements need updating.

The rows are organized around the states with the first three rows
indicating missing entries for the "raw" state the would appear in
the first row, or missing elements in the "edited" state or "good"
state that exist in the "raw" state in the second and third.  The
fourth through sixth rows are comparisons to find which elements
don't match.  The fourth row being if the "raw" state elements and
the "edited" state elements don't match. The fifth a comparison
between the "raw" state and "good" state.  The sixth being between
the "edited" state and the "good" state. Then, the final row is for
if everything across all the states match each other.  The goal is
to achieve everything listed in the final row.

The columns are a combination of the various languages available
and the types and destinations of each element.

To bring up the list of individuals in any particular category of
missing, incorrect, or even correct information, simply click on
the number of elements in the state you wish to work with.  In the
example below, there are seven individuals who do not have an
edited en-us web entry in the biographical information matrix.  By
clicking on the "7" in that section of the table, you bring up the
list of participants who's elements need editing.

The table might resemble the following:
\begin{table}[htb]
\caption{\label{tbl:staffmanageparticipantbiographies}Staff - Manage Participant Biographies}
\centering
\begin{tabular}{lrrrrrrrr}
\hline
Count of the States of the bios & en-us name web & fr-ca name web & en-us name book & fr-ca name book & en-us web uri & fr-ca web uri & en-us book uri & fr-ca book uri\\
\hline
Missing raw bio &  &  & 3 & 6 & 59 & 61 & 14 & 61\\
Missing edited bio & 7 & 16 & 21 & 5 & 48 & 61 & 14 & 61\\
Missing good bio & 59 & 59 & 59 & 59 & 59 & 61 & 14 & 61\\
Raw bio doesn't match edited bio & 17 & 11 & 18 & 19 & 9 &  &  & \\
Edited bio doesn't match good bio & 12 & 9 & 28 & 46 & 9 &  &  & \\
All bios match & 5 & 3 & 4 &  &  & 61 & 47 & 61\\
\hline
\end{tabular}
\end{table}

Once the particular subsection for editing has been picked, you
will be provided with a list of names in this category, so you
might choose the one(s) to be edited.  Illustrated below.

The three columns in this table are:
\begin{itemize}
\item Participant: Each name in this column is a link that will bring
you to the page referenced by section \ref{sec-3-2} for the
individual chosen, so the "edited" state can be updated.  Also
from here is where the information might be promoted to the
"good" state if everything is in agreement.
\item Edit Full: Each name in this column is a link that will bring you
to the page referenced by section \ref{sec-3-3}
to edit the "raw" state of the information on the participant
chosen. This is the full information we have on an individual, in
case that such information is necessary to find out more about
who they are, or if there are any notes, or the like to assist
you with the biographical information editing.
\item Currently being edited by: This column indicates the individual
who is the one editing the particular individual's information at
this time.  Please, do not choose to edit the biographical
information of someone who is locked by someone other than you.
\end{itemize}

The table might resemble the following:
\begin{table}[htb]
\caption{\label{tbl:staffmanageparticipantbiographiessubedit}Staff - Manage Participant Biographies Subedit}
\centering
\begin{tabular}{lll}
\hline
Participant & Edit Full & Currently being edited by\\
\hline
Mr E. & Mr E. & \\
Joker & Joker & Str8mn\\
Batman & Batman & The Riddler\\
The Riddler & The Riddler & Batman\\
Catwoman & Catwoman & \\
Cartman & Cartman & \\
Robin & Robin & \\
\hline
\end{tabular}
\end{table}

There is also the "return" link, just above the table, that brings
you back to the page referenced by the table \hyperref[tbl:staffmanageparticipantbiographies]{"Staff - Manage
Participant Biographies"}, in case you are done with this particular
subset of individuals, and wish to deal with others.

\subsection{StaffEditBios.php}
\label{sec-3-2}
\label{StaffEditBios.php}
\underline{
\href{../webpages/StaffEditBios.php}{StaffEditBios.php}
}

This page is where an individual's information is edited from the
point of view of the con staff.  Once this page is opened for a
participant, the editing individual's name will be placed in the
"Currently being edited by" column, so two people don't try to edit
the same bio information at the same time.

There are two links at the top of the page.  The first is the
individuals name and will lead you to the page mentioned section
\ref{sec-3-3}, in the mode of editing the
participant chosen, and the return link, which will bring you back
to the table above on the page from section \ref{sec-3-1}.

Any of the "Save Whole Page" buttons will save the current state of
the entire page, and clear the "Currently being edited by" flag.

The "Promote \ldots{} to good." links will only show up if the "raw"
state information matches the "edited" state information.

The "Good entry exists for \ldots{} Biography" informational statement
is there to let you know that the information has been promoted all
the way up to the "good" state at least once before.

The design of this page has a series of elements, alternating "raw"
state and "edited" state boxes, with the "raw" state boxes there
for copying purposes, for they are not malleable.  Which entries
are available are dependent on what access the individual has.  The
ordering of the elements on the page will be something like:
\begin{itemize}
\item raw uri book en-us
\item edited uri book en-us
\item raw uri book fr-ca
\item edited uri book fr-ca
\item raw uri web en-us
\item edited uri web en-us
\item raw uri web fr-ca
\item edited uri web fr-ca
\item raw bio book en-us
\item edited bio book en-us
\item \ldots{}
\end{itemize}

There might be style guide for the formatting of the information in
each of the informational entries.  Things like:
\begin{itemize}
\item Each of the web and book bios entries are pre-pended
automatically with the web and book name entries, bios should be
written in the third person, and start with a space character.
\item Multi-entry web uri entries should be separated by a web-style
line-break.
\item Multi-entry book uri entries should be separated by a
double-colon
\end{itemize}

The database is (by default) a Latin-8 character set, so anything
in the entries that is outside of that character set, will, most
probably either be replaced by "?" or break the entry.

Entries into the blocks can be as long as you like, but, they will
be length-checked, and rejected if they are longer than the limits
set by the PublicationLimits table.

\subsection{StaffEditCreateParticipant.php}
\label{sec-3-3}
\label{StaffEditCreateParticipant.php}
\underline{
\href{../webpages/StaffEditCreateParticipant.php?action=edit}{StaffEditCreateParticipant.php}
}

While this page is of more general use (which is more fully
documented in the Presenter Flow Document) if absolutely necessary
it can be used to modify the raw biographical information. The raw
state of all of the types (uri, facebook, bio, etc.), destinations
(web, book, badge, etc.), and languages are available to be edited
on this page.
\end{document}