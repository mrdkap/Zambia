\documentclass[captions=tablesignature]{scrartcl}
\usepackage[utf8]{inputenc}
\usepackage[T1]{fontenc}
\usepackage{fixltx2e}
\usepackage{graphicx}
\usepackage{longtable}
\usepackage{float}
\usepackage{wrapfig}
\usepackage{rotating}
\usepackage[normalem]{ulem}
\usepackage{amsmath}
\usepackage{textcomp}
\usepackage{marvosym}
\usepackage{wasysym}
\usepackage{amssymb}
\usepackage{hyperref}
\tolerance=1000
\usepackage{booktabs}
\usepackage[scaled]{beraserif}
\usepackage[scaled]{berasans}
\usepackage[scaled]{beramono}
\usepackage[usenames,dvipsnames]{color}
\usepackage{fancyhdr}
\usepackage{subfig}
\usepackage{listings}
\lstnewenvironment{common-lispcode}
{\lstset{language={HTML},basicstyle={\ttfamily\footnotesize},frame=single,breaklines=true}}
{}
\usepackage{paralist}
\let\itemize\compactitem
\let\description\compactdesc
\let\enumerate\compactenum
\usepackage[letterpaper,includeheadfoot,top=12.5mm,bottom=25mm,left=19mm,right=19mm]{geometry}
\pagestyle{fancy}
\setcounter{secnumdepth}{3}
\author{Percy\thanks{NELA.Percy@gmail.com}}
\date{March 2012}
\title{Panelists Guide for Zambia}
\hypersetup{
  pdfkeywords={Zambia, Documentation, FFF branch},
  pdfsubject={Zambia is a piece of Conference Management Software.  This document is a "How To" guide assisting panelists in the use of the Zambia FFF-branch instance for your conference.  This is still a work in progress.},
  pdfcreator={}}
\begin{document}

\maketitle
\pagenumbering{roman}
\thispagestyle{fancy}
\renewcommand{\headrulewidth}{0pt}
\renewcommand{\footrulewidth}{0pt}
\lhead{}
\rhead{}
\chead{}
\lfoot{}
\cfoot{}
\rfoot{}
\begin{abstract}
\vspace{5cm}
{\LARGE{\textbf{Abstract:\\}}}
Zambia is a piece of Conference Management Software.  This document is a "How To" guide assisting panelists in the use of the Zambia FFF-branch instance for your conference.  This is still a work in progress.
\end{abstract}
\newpage
\renewcommand{\headrulewidth}{1pt}
\renewcommand{\footrulewidth}{1pt}
\chead{
Panelists Guide for Zambia
}
\lfoot{
Percy <NELA.Percy@gmail.com>
}
\rfoot{\thepage}
\setcounter{tocdepth}{3}
\tableofcontents
\newpage
\pagenumbering{arabic}
\section{Introduction}
\label{sec-1}

Zambia is the system used by this convention to organize and
schedule our content.  Logging into Zambia as a participant allows
panelists to indicate which panels they would like to be on, set
their availability during the convention and create a
profile/biography.

\section{First login/Welcome}
\label{sec-2}
\label{welcome.php}
\underline{
\href{../webpages/welcome.php}{Welcome}
}

The first page you see, when you log in will have a lot of
important, and general information on it.  Please read it, and let
your liaisons know if you have any issues.
\section{My Profile}
\label{sec-3}
\label{my_contact.php}
\underline{
\href{../webpages/my_contact.php}{My Profile}
}

When you log in for the first time, you will need to update your
profile and change your password away from the default, or the
password sent upon reset. This can be done by clicking "profile" on the menu bar.

From this page you can do the following:
\begin{itemize}
\item Let us know if you are interested and able to participate in
programming for the upcoming convention:
\begin{itemize}
\item If you want to be on panels or otherwise be a program
participant, set it to "Yes".
\item If you are unable to attend this event, but want to be invited
for future events, please still log in and set it to "No".
\end{itemize}
\item Let us know if we can share your e-mail address with other
participants.  Please consider saying yes, so that moderators of
the panels you have been selected for can contact you prior to the
convention to let you know their format, so you can discuss and
prep for the panel.  Saying no will not affect our decisions when
making panelist selections for the panels.
\item Let us know if we can photograph you while you are participating
on the panels, and if we can use those images in the promotion of
the convention.  Saying no will not affect our decisions when
making panelist selections for panels.
\item Let us know how you want your name and other bio information to
appear in our electronic and print publications.  If you have
questions about this, please contact your liaison.  This
information will be presented, with what we currently have for
you, and gives you space to change it.  When and if you change
your bios information, either your liaison, or a bios editor might
be in touch, if there is any further changes we might suggest.
\item Review the contact information we have for you.
\end{itemize}

\section{My Availability}
\label{sec-4}
\label{my_sched_constr.php}
\underline{
\href{../webpages/my_sched_constr.php}{My Availability}
}

Next, please let us know about the times you are available to be on
panels, and the number of panels you are willing to be on, by
selecting the "availability" option from the menu in the page
header. This allows us to get an idea of what kind of schedule you
would like to have during the convention. In addition to listing
time ranges, make sure to enter information on any constraints or
conflicts that might occur on your schedule at the bottom of the
page. Be as thorough as you can and fill out all fields on this page
-- the more information we have, the better! Please keep in mind
that a narrow availability listing makes it difficult for our staff
to schedule you, so please be both honest and realistic about your
con participation time!

\section{Search Panels}
\label{sec-5}
\label{my_sessions1.php}
\underline{
\href{../webpages/my_sessions1.php}{Search Panels}
}

Next up is "Search Panels"; this is how you find out what sort of
panels and other program items are available for
participation. Because of the volume of selections that are
available at this convention, we organize panels by tracks. There
are no restrictions on how many tracks a person can participate on;
we love panelists that have knowledge in several content areas and
can participate on a diverse range of items!

From the "Search Panels" page you can do the following:
\begin{itemize}
\item Click the drop menu on the "Track" option. This will show you a
list of all our tracks or if there is a specific panel number that
can be selected as well.
\item Once you have made your selections, click "Search". This will give
you a list of panel options that match your selection.
\item Each entry has a box next to the title that you click on to add it
as a panel you are interested in participating in.
\item Once you have made your selections, click "Save". You will be taken
to a page where you can enter a rating and give reasons as to why
you would be a good choice for a given panel (see the section
\ref{sec-6} below).
\item Go back to the "Search Panels" page to add more panels to your
list, either from the same track, or other tracks.
\end{itemize}

\section{My Panel Interests}
\label{sec-6}
\label{PartPanelInterests.php}
\underline{
\href{../webpages/PartPanelInterests.php}{My Panel Interests}
}

Once you have selected a bunch of panels you are interested in
participating in, it is time to provide us with information for why
we should place YOU on that panel.  This is a really important step!
We have so many great panelists that sign up each event, so we rarely
assign people to panels who have not ranked the panel and given
reason as to why they would be a fabulous addition to this panel.

Click on the "My Panel Interests" option from the menu to view all
the panels you have previously selected.
\begin{itemize}
\item First, go through and rank all your panel selections. We use the
following rating system:
\begin{itemize}
\item 1 -- Oooh! Oh! Pick Me!
\item 2-3 -- I'd like to if I can.
\item 4-5 -- I am qualified but this is not one of my primary
interests.
\end{itemize}
You are limited to 4 sessions each of preferences 1-4 and there is
no limit to the number of sessions for which you can express
preference 5.
\item Next, for each panel, write in the text boxes to let us know
why you would be fabulous on that session.
\begin{itemize}
\item Completing this step thoughtfully is especially important for
the items you rank highest.
\item Address anything asked of panelists in the beige box associated
with that panel.
\item This does not need to be long -- a short paragraph is typical.
\item Some panelists draft these out in a separate document and then
cut-and-paste their entries into Zambia as a final step.
\item If you do not fill this section out, you might not get placed on
the panel, even if you rate it a 1 and feel that you are an
obvious choice.
\end{itemize}
\item If you are interested in moderating a particular panel, please
select that option while providing your rankings.  The panelists
most enthusiastic about a topic do not always make the best
moderators, as moderators are facilitators in addition to
contributors on a panel. So keep this in mind when ranking and
indicating your interest in moderating.
\item Remember to save your progress approximately every ten minutes, so
that you are not logged out due to inactivity and lose your
selections or what you wrote!
\item Ranking a panel as "1" and writing a thesis on why you are perfect
for the panel does not guarantee your placement on it. Most of our
panels have more qualified people sign up for them than we have
spaces on the panel (typically 4-5), and there are more sessions
to sign up for than we have spots in the schedule.
\end{itemize}

\section{My General Interests}
\label{sec-7}
\label{my_interests.php}
\underline{
\href{../webpages/my_interests.php}{My General Interests}
}

The next section you will want fill out is "General Interests". This
page is where you can provide information pertaining to your
interests, workshops or presentations you would like to pitch to us,
who you would like to be on panels with, and/or who you want to
avoid.  This is also the place to give us other information we
should consider when creating this event's programming schedule.

Some people find that it is best to fill out this section last,
after completing all other parts of the panel sign-up process.  You
should feel free to come back to it at the end.

\section{My Schedule}
\label{sec-8}
\label{MySchedule.php}
\underline{
\href{../webpages/MySchedule.php}{My Schedule}
}

This page gives you your schedule for this event followed by your
schedule and feedback from previous events.  It also has links for
an ical (for your phone/tablet/computer calendar) and a printable
schedule.  Your schedule should also be in your panelist packet.

\section{Contact}
\label{sec-9}

There should be a contact email at the bottom of each page, if you
need help or assistance.  Also, please reach out to your liaison at
any time if you need more information, or anything.
\end{document}