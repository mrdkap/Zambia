\documentclass[captions=tablesignature]{scrartcl}
\usepackage[utf8]{inputenc}
\usepackage[T1]{fontenc}
\usepackage{fixltx2e}
\usepackage{graphicx}
\usepackage{longtable}
\usepackage{float}
\usepackage{wrapfig}
\usepackage{rotating}
\usepackage[normalem]{ulem}
\usepackage{amsmath}
\usepackage{textcomp}
\usepackage{marvosym}
\usepackage{wasysym}
\usepackage{amssymb}
\usepackage{hyperref}
\tolerance=1000
\usepackage{booktabs}
\usepackage[scaled]{beraserif}
\usepackage[scaled]{berasans}
\usepackage[scaled]{beramono}
\usepackage[usenames,dvipsnames]{color}
\usepackage{fancyhdr}
\usepackage{subfig}
\usepackage{listings}
\lstnewenvironment{common-lispcode}
{\lstset{language={HTML},basicstyle={\ttfamily\footnotesize},frame=single,breaklines=true}}
{}
\usepackage{paralist}
\let\itemize\compactitem
\let\description\compactdesc
\let\enumerate\compactenum
\usepackage[letterpaper,includeheadfoot,top=12.5mm,bottom=25mm,left=19mm,right=19mm]{geometry}
\pagestyle{fancy}
\setcounter{secnumdepth}{3}
\author{Percy\thanks{NELA.Percy@gmail.com}}
\date{January 2016}
\title{Photo Lounge Flow in Zambia}
\hypersetup{
  pdfkeywords={Zambia, Documentation, FFF branch},
  pdfsubject={Zambia is a piece of Conference Management Software.  This document is a "How To" guide assisting in the way of keeping track of Volunteer Hours for the Zambia FFF-branch instance for your conference.  This is still a work in progress.},
  pdfcreator={}}
\begin{document}

\maketitle
\pagenumbering{roman}
\thispagestyle{fancy}
\renewcommand{\headrulewidth}{0pt}
\renewcommand{\footrulewidth}{0pt}
\lhead{}
\rhead{}
\chead{}
\lfoot{}
\cfoot{}
\rfoot{}
\begin{abstract}
\vspace{5cm}
{\LARGE{\textbf{Abstract:\\}}}
Zambia is a piece of Conference Management Software.  This document is a "How To" guide assisting in the way of keeping track of Volunteer Hours for the Zambia FFF-branch instance for your conference.  This is still a work in progress.
\end{abstract}
\newpage
\renewcommand{\headrulewidth}{1pt}
\renewcommand{\footrulewidth}{1pt}
\chead{
Photo Lounge Flow in Zambia
}
\lfoot{
Percy <NELA.Percy@gmail.com>
}
\rfoot{\thepage}
\setcounter{tocdepth}{3}
\tableofcontents
\newpage
\pagenumbering{arabic}
\section{Introduction}
\label{sec-1}

The Photo Lounge is a complex addition to Zambia that might not be
used by all groups.  Some groups have used it, some have not.  Like
all optional pieces, this can simply be untouched, and un-setup, and
it will not bother anyone.  On the other hand, if you wish to use
it, and have the ability to host something like a Photo Lounge or
Art Gallery or the like, this is the right place to be doing so.
\section{Basic Setup}
\label{sec-2}

Like all sub-sections of Zambia, to activate this you have to make
sure you have a number of elements in your con setup in place.
\subsection{PhaseTypes}
\label{sec-2-1}

A pair of PhaseTypes included in your con.  They would be "Photo
Voting (17)" and "Photo Submissions (18)".

Toggling the Photo Submission Phase, adds and removes the inclusion
of the link to propose yourself as a photo submitter (a public
page, which writes to a file, we'll get to that later) and your
standard login to include the photo submitting words in the login
string.  It also allows for the photo submission pages to be
active.

Toggling the Photo Voting Phase, adds and remvoes the links or
access to voting for those who are of the right permissions to
vote.

\subsection{PermissionRoles}
\label{sec-2-2}

The three permission roles associated with the Photo Lounge are:
\begin{itemize}
\item PhotoSub
\item PhotoRev
\item SuperPhotoRev
\end{itemize}

PhotoSub is pretty straight-forward.  It is for those who are
allowed to submit photos.  If they have no other role, then their
login will take them directly to the Photo Submissions page.  If
they do have other roles with the event (like presenting or
volunteering) it will add the tab to their tab bar.  If they have
other super-roles, then, to access the Photo Submission tab, they
have to be under the "Participant View" section of the website to
be able to submit their photos.

PhotoRev is for those folks trusted with the job of reviewing
photos.  It should open to the appropriate welcome page, and have
the Photo Review tab available to them.  That leads to the voting
section so they can express their opinions on what might or might
not be included in the Photo Lounge.  There are still some quirks
around voting, and one can game the system, by voting multiple
times.  This will have to be coded around at some point in the
future.

SuperPhotoRev allows one to be able to select the voted on pictures
for the lounge, give people permission to be PhotoRev and PhotoSub,
and has StaffView.  This might want to become more limited, if the
Super doesn't have other powers, to limit their access and review.
Also the Super should be able to toggle on and off the Phases, but
that also is still a work in progress.

\subsection{ConRoles}
\label{sec-2-3}

There is one con role that works with this, but that is the host of
the Photography Lounge, so not directly relevant to this document.
See the document on Lounge Hosts.

\section{Scripts}
\label{sec-3}

There are two scripts that are useful/necessary to the well-running
of the Photo Lounge Flow.  They are:
\begin{itemize}
\item create\_thumbnails.sh
\item create\_imagination\_files.sh
\end{itemize}

The create\_thumbnails.sh script does just that.  Creates the
thumbnail files.  It should be run periodically on a host that has
access to the files, and can run convert from the imagemagick suite
of programs to make the smaller, and much more useful, thumbnail
files.

The create\_imagination\_files.sh script is a script that will pull
all the accepted files into a directory so that the imagination
software from \url{http://imagination.sourceforge.net/} can be used to
create the slideshow.

Both are fairly straight-forward and simple, but useful.

\section{Files}
\label{sec-4}

There is a possiblity you would want to tweak the values at the top
of the PhotoLoungeSubmit.php file.  They are:
\begin{itemize}
\item \$minwidth=1200;
\item \$minheight=1200;
\item \$maxwidth=30000;
\item \$maxheight=30000;
\item \$maxfilesize=6000000;
\end{itemize}

and they map to, the minimum width, minumum height (both must
be more than), maxumum width, maximum height (both must be less
than) and maximum file size (must be less than).

\section{Flow}
\label{sec-5}

\begin{itemize}
\item The flow starts when the call for Photo Submissions begins.
\item The SuperPhotoRev either asks for the phase to be set, or sets the
phase to open, so that people might either submit, or apply to
submit.
\item If self-service selection is enabled, then anyone who has been a
Photo Submitter before can self-select into having those
permissions again, on their own.
\item Anyone can be migrated to be allowed to be a Photo Submitter,
again by the SuperPhotoRev.
\item If it is desired to just set everyone who has before been a Photo
Submitter, to have that bit set for this event (replacing \$conid
with the current conid, of course):
\end{itemize}
\begin{verbatim}
SELECT
    DISTINCT concat("INSERT INTO UserHasPermissionRole VALUES (",badgeid,",41,$conid);")
  FROM
      UserHasPermissionRole
  WHERE
    permroleid=41;
\end{verbatim}
\begin{itemize}
\item The PhotoSubs then submit photos at the \href{../webpages/PhotoLoungeSubmit.php}{PhotoLoungeSubmit.php}
page.
\item Once the photos are all submitted or the SuperPhotoRev is done
soliciting them, that phase should be closed, and the Photo Voting
phase should be initiated.
\item This would be a good time to have the create\_thumbnails.sh script
run.  That will make sure that all the thumbnails are in place,
before the next step.
\item Folks that either have self-selected, if self-service has been
enabled, or have been selected by the SuperPhotoRev may now go and
vote on all the pictures at the \href{../webpages/PhotoLoungePictures.php}{PhotoLoungePictures.php} page
or by going directly to the individual artist selections at the
\href{../webpages/PhotoLoungeVote.php}{PhotoLoungeVote.php} page.
\item Once sufficient pictures have been voted on, the voting phase can
close.
\item The selection process, of the pictures that are voted on have
several steps, some of which can go in various orders, to be
determined by the SuperPhotoRev.
\item View and select which of the voted-upon sets of pictures should be
included into the lounge at the \href{../webpages/PhotoLoungeCollectVotes.php}{PhotoLoungeCollectVotes.php} page.
\item Using the contact sheet, get in touch with all the appropriate
PhotoSub folks to let them know what was selected from the
\href{../webpages/PhotoLoungeContactSheet.php}{PhotoLoungeContactSheet.php} page.
\item If there are any changes, they can be managed from:
\begin{itemize}
\item the same contact sheet - \href{../webpages/PhotoLoungeContactSheet.php}{PhotoLoungeContactSheet.php}
\item the view of the voted on pictures - \href{../webpages/PhotoLoungeCollectVotes.php}{PhotoLoungeCollectVotes.php}
\item the report that also has the same information, without the
possible confusion of changing what was selected -
\href{../webpages/genreport.php?reportname=picturesubmiswvote}{genreport.php?reportname=picturesubmiswvote}
\end{itemize}
\item Once the photos are all selected, this is a good time to run the
create\_imagination\_files.sh script so that there is a directory to
pull the appropriate files from, and create the slideshow from.
\end{itemize}

\section{Using the imagination software}
\label{sec-6}

\begin{itemize}
\item Create a new slideshow: VOB/576 HD, no rescale, background=black
\item Import the images from the directory created by the
create\_imagination\_files.sh script (probably
Local/\$conid/Photo\_Lounge\_Accepted)
\item Select all the images, and set the slide duration to 5.
\item Select all the images, and set a random transition, to be
modified.
\item Select the speed (4)
\item Go through and fix all the transitions
\item Add the appropriate music
\item Export to a VOB file
\end{itemize}
\subsection{File exerpt:}
\label{sec-6-1}

\begin{verbatim}
[slideshow settings]
video format=576
background color=0;0;0;
distort images=false
number of slides=75

[slide 1]
filename=/home/dkap/hcarperoot/Downloads/PhotoLounge/Anthelian_1_mg_7182a.jpg
angle=0
duration=5
transition_id=13
speed=4
no_points=0
anim id=0
anim duration=1
text pos=4
placing=0
font=Sans 12
font color=0;0;0;1;
font bgcolor=1;1;1;1;
\end{verbatim}

\section{Convert VOB to mpeg/avi}
\label{sec-7}

\begin{itemize}
\item ffmpeg -i FFF??\_Photo\_Lounge.vob -f avi -c:v mpeg4 -b:v 800k -g 300 -bf 2 -c:a libmp3lame -b:a 128k FFF??\_Photo\_Lounge.avi
\item cp FFF??\_Photo\_Lounge.vob FFF??\_Photo\_Lounge.mpeg
\end{itemize}

\section{Load onto Devices}
\label{sec-8}

\begin{itemize}
\item Thumb Drives:  Pictures into Pictures, only the mpeg in Movies
\item Pi Devices: Everthing into local Pictures, mpeg into sub Movie diretory in Pictures.
\end{itemize}
\end{document}