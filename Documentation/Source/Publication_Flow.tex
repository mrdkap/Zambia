\documentclass[captions=tablesignature]{scrartcl}
\usepackage[utf8]{inputenc}
\usepackage[T1]{fontenc}
\usepackage{fixltx2e}
\usepackage{graphicx}
\usepackage{longtable}
\usepackage{float}
\usepackage{wrapfig}
\usepackage{rotating}
\usepackage[normalem]{ulem}
\usepackage{amsmath}
\usepackage{textcomp}
\usepackage{marvosym}
\usepackage{wasysym}
\usepackage{amssymb}
\usepackage{hyperref}
\tolerance=1000
\usepackage{booktabs}
\usepackage[scaled]{beraserif}
\usepackage[scaled]{berasans}
\usepackage[scaled]{beramono}
\usepackage[usenames,dvipsnames]{color}
\usepackage{fancyhdr}
\usepackage{subfig}
\usepackage{listings}
\lstnewenvironment{common-lispcode}
{\lstset{language={HTML},basicstyle={\ttfamily\footnotesize},frame=single,breaklines=true}}
{}
\usepackage{paralist}
\let\itemize\compactitem
\let\description\compactdesc
\let\enumerate\compactenum
\usepackage[letterpaper,includeheadfoot,top=12.5mm,bottom=25mm,left=19mm,right=19mm]{geometry}
\pagestyle{fancy}
\setcounter{secnumdepth}{3}
\author{Percy\thanks{NELA.Percy@gmail.com}}
\date{March 2012}
\title{Presenter Flow in Zambia}
\hypersetup{
  pdfkeywords={Zambia, Documentation, FFF branch},
  pdfsubject={Zambia is a piece of Conference Management Software.  This document is a "How To" guide assisting in a path of pulling the publications out of the Zambia FFF-branch instance for your conference.  This is still a work in progress.},
  pdfcreator={}}
\begin{document}

\maketitle
\pagenumbering{roman}
\thispagestyle{fancy}
\renewcommand{\headrulewidth}{0pt}
\renewcommand{\footrulewidth}{0pt}
\lhead{}
\rhead{}
\chead{}
\lfoot{}
\cfoot{}
\rfoot{}
\begin{abstract}
\vspace{5cm}
{\LARGE{\textbf{Abstract:\\}}}
Zambia is a piece of Conference Management Software.  This document is a "How To" guide assisting in a path of pulling the publications out of the Zambia FFF-branch instance for your conference.  This is still a work in progress.
\end{abstract}
\newpage
\renewcommand{\headrulewidth}{1pt}
\renewcommand{\footrulewidth}{1pt}
\chead{
Presenter Flow in Zambia
}
\lfoot{
Percy <NELA.Percy@gmail.com>
}
\rfoot{\thepage}
\setcounter{tocdepth}{3}
\tableofcontents
\listoffigures
\listoftables
\newpage
\pagenumbering{arabic}
\section{Introduction}
\label{sec-1}

There is a certain pattern to the flow of dealing with the
publications, with an eye mostly to the concept of a book or
pamphelet as a hard-copy guide available for your con.  There are
several pieces to this, from ads and sponsors, to the vending and
presenter data that go into that publication, but this also
describes things like the presenter, volunteer, vendor, and general
attendee packets as well.
\section{General Flow}
\label{sec-2}
\subsection{Set Sponsorship levels}
\label{sec-2-1}
Go to the \href{../webpages/PubsSetupAds.php}{PubsSetupAds.php} page, and set the
appropriate sponsor levels available/prices/et al for this event.

\subsection{Set Digital Ad possiblities}
\label{sec-2-2}
Go to the \href{../webpages/PubsSetupAds.php}{PubsSetupAds.php} page, and set the
appropriate digital ads available/prices/et al for this event.

\subsection{Set Printed Ad possibilities}
\label{sec-2-3}
Go to the \href{../webpages/PubsSetupAds.php}{PubsSetupAds.php} page, and set the
appropriate print ads available/prices/et al for this event.
\subsection{Task List}
\label{sec-2-4}

Create \href{../webpages/TaskListUpdate.php?activityid=-1}{"Task List"} entries to track timing across the event.

Things to consider: 
\begin{itemize}
\item informational deadlines as editing is needed
\item book/pamphelet deadlines
\item handouts deadlines, printing (including count), and distributing
\item packet information deadlines and contents
\item signage deadlines and placement timing
\item Social Media deadlines, rolling and fixed
\item BEO deadlines
\item Feedback Form deadlines - printing, collecting, input, and
reporting
\item Sponsor Banner deadlines
\item Sponsor Class deadlines (including negotiation times with the
class presenters)
\end{itemize}

\subsection{PublicationLimits}
\label{sec-2-5}

Set up the publication limits for the data collected on the
presenters, session elements, vendors, and volunteers.  This is
described in more depth in the \texttt{"Setting Up"} \texttt{(PDF)} document.

\section{Program Book/Pamphelet}
\label{sec-3}

Most of the programming book elements can be found in the \href{../webpages/PreconPrinting.php}{"Printing"}
page:
\begin{itemize}
\item \href{../webpages/BookBios.php}{Bios} \href{../webpages/BookBios.php?pic_p=N}{(without images)} \href{../webpages/BookBios.php?short=Y}{(short)}
\item \href{../webpages/BookStaffBios.php}{Staff Bios} \href{../webpages/BookStaffBios.php?pic_p=N}{(without images)} \href{../webpages/BookStaffBios.php?short=Y}{(short)}
\item \href{../webpages/BookSched.php?format=desc}{Descriptions} \href{../webpages/BookSched.php?format=desc&short=Y}{(short)}
\item \href{../webpages/BookSched.php?format=sched}{Schedule} \href{../webpages/BookSched.php?format=sched&short=Y}{(short)}
\item \href{../webpages/BookSched.php?format=tracks}{Track list by Name} \href{../webpages/BookSched.php?format=tracks&short=N}{(short)}
\item \href{../webpages/BookSched.php?format=trtime}{Track list by Time} \href{../webpages/BookSched.php?format=trtime&short=Y}{(short)}
\item \href{../webpages/BookSched.php?format=rooms}{Rooms} \href{../webpages/BookSched.php?format=rooms&short=Y}{(short)}
\item If written the letter from the con chair(s), the board chair(s),
the rules, the FAQ, maps, etc are all in the Local/conid/
directory for inclusion e.g. Local/2/FAQ
\item Ads are still done elsewhere
\end{itemize}

Not always is all the information in Zambia by the time it is being
collected to print, so \ldots{} if there are tags like: \textbf{\textbf{\textbf{EDIT PLEASE}}}
that usually means that some information is missing.  Please see the
\texttt{"Bios Editing"} \texttt{(PDF)} document for further instructions.  Also if a
picture is listed as: Picture for editing at: (path)/Local/logo.gif
it usually means there isn't a picture available.

\section{Presenter, Volunteer, Class, and Vendor Packets}
\label{sec-4}

Much of what goes into the packets are the same or simiar across
several of them.  There should be at least one "Packet Stuffing"
Task that has the details in it.
\section{Photo Lounge Contact Sheet}
\label{sec-5}

This is available on the \href{../webpages/PhotoLoungeContactSheet.php}{Photo Lounge Contact Sheet} page.

This usually has two or three laminated copies printed and then
punched on a ring, so people can look, but they get the impression
that they are not there for the taking.  At one event, a pad of
sticky notes and a pen were also attached to the ring, so that if
someone wanted to note down a particular photo, or artist for
getting prints from, or to do work with, that was available.

\section{Room Logistics}
\label{sec-6}

The \href{../webpages/LogisticsPrint.php}{"Room States"} and the \href{../webpages/genreport.php?reportid=188}{"Beo Form"} are a good place to start
working on printing out the BEOs so they can be checked, turned in
to the hotel and then signed off on, with the hotel.  It is often
useful to check the information first on a few reports like the
\href{../webpages/genreport.php?reportid=137}{"Combined Roomsets for Programming"}, or the \href{../webpages/genreport.php?reportid=100}{"Combined Roomset"}.

\section{Social Media Spreadsheet}
\label{sec-7}

This spreadsheet \href{../webpages/SocialMediaSpreadsheet.php}{"SocialMediaSpreadsheet.php"} has sections for each
of the types of information that might want to be disseminated.
There are a bunch of sections, from presenters to community tables,
etc.

\section{Grids}
\label{sec-8}

The grids are automatically produced from the database in several
different ways.  There is the one for \href{../webpages/KonOpas.php#info}{"KonOpas"}, the printable/live
ones that are \href{../webpages/Postgrid-wide.php}{"Times x Rooms"} \href{../webpages/Postgrid-wide.php?print_p=y}{(PDF)} or \href{../webpages/Postgrid.php}{"Rooms x Times"} \href{../webpages/Postgrid.php?print_p=y}{(PDF)} and a
\href{../webpages/manualGRIDS.php}{bunch of others}.

\section{Feedback}
\label{sec-9}

Set up the feedback forms, for online and publication purposes.
This allows the feedback on the classes to be collected, entered,
and tracked, both for the presenter's benefit (to help them improve
their classes) and for the program committee's benefit (to help them
both select who from the past, what classes landed and what didn't
and what size rooms and timing is better for what).  This is
described in more depth in the \texttt{"Setting Up"} \texttt{(PDF)} document.

\section{Feedback Forms}
\label{sec-10}

These should be available on the \href{../webpages/StaffFeedback.php}{"Feedback forms"} page once the
above procedure is followed appropriately.
\end{document}