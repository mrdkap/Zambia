\documentclass[captions=tablesignature]{scrartcl}
\usepackage[utf8]{inputenc}
\usepackage[T1]{fontenc}
\usepackage{fixltx2e}
\usepackage{graphicx}
\usepackage{longtable}
\usepackage{float}
\usepackage{wrapfig}
\usepackage{rotating}
\usepackage[normalem]{ulem}
\usepackage{amsmath}
\usepackage{textcomp}
\usepackage{marvosym}
\usepackage{wasysym}
\usepackage{amssymb}
\usepackage{hyperref}
\tolerance=1000
\usepackage{booktabs}
\usepackage[scaled]{beraserif}
\usepackage[scaled]{berasans}
\usepackage[scaled]{beramono}
\usepackage[usenames,dvipsnames]{color}
\usepackage{fancyhdr}
\usepackage{subfig}
\usepackage{listings}
\lstnewenvironment{common-lispcode}
{\lstset{language={HTML},basicstyle={\ttfamily\footnotesize},frame=single,breaklines=true}}
{}
\usepackage{paralist}
\let\itemize\compactitem
\let\description\compactdesc
\let\enumerate\compactenum
\usepackage[letterpaper,includeheadfoot,top=12.5mm,bottom=25mm,left=19mm,right=19mm]{geometry}
\pagestyle{fancy}
\setcounter{secnumdepth}{3}
\author{Percy, Amanda, Rob, Iya\thanks{NELA.Percy@gmail.com}}
\date{May 2017}
\title{Vending Design Document}
\hypersetup{
  pdfkeywords={Zambia, Documentation, FFF branch},
  pdfsubject={Zambia is a piece of Con Management Software.  This document is a guide to building the Vendor System for the Zambia FFF-branch instance.  This is still a work in progress.},
  pdfcreator={}}
\begin{document}

\maketitle
\pagenumbering{roman}
\thispagestyle{fancy}
\renewcommand{\headrulewidth}{0pt}
\renewcommand{\footrulewidth}{0pt}
\lhead{}
\rhead{}
\chead{}
\lfoot{}
\cfoot{}
\rfoot{}
\begin{abstract}
\vspace{5cm}
{\LARGE{\textbf{Abstract:\\}}}
Zambia is a piece of Con Management Software.  This document is a guide to building the Vendor System for the Zambia FFF-branch instance.  This is still a work in progress.
\end{abstract}
\newpage
\renewcommand{\headrulewidth}{1pt}
\renewcommand{\footrulewidth}{1pt}
\chead{
Vending Design Document
}
\lfoot{
Percy, Amanda, Rob, Iya <NELA.Percy@gmail.com>
}
\rfoot{\thepage}
\setcounter{tocdepth}{2}
\tableofcontents
\listoftables
\listoffigures
\newpage
\pagenumbering{arabic}
\section{General Design}
\label{sec-1}
This is a working guide on how the vending portion of Zambia will be
designed/changed.  Some of it is already in place, some has yet to
be built.  This guide should have all the relevant bits in it so
that all the stake-holders can have their say, and we are working
off of one document.

NOTE: when there is a - or a + next to things, it's because I asked
that as a question, or made a supposition, and it was answered (by
Amanda) so, for now at least the - is my side of the conversation
and the + is her side of the conversation.

\section{Database Design}
\label{sec-2}
\subsection{Perennial}
\label{sec-2-1}
These are the current values that should be migrated to the Zambia
values if they are not already set.  The latest entries will be
considered canonical.  The ones of these in CongoDump will be more
or less fixed and (currently) take an act of Vendor Coordinator (or
other back-end Zambia person) to update.  

The ones of these in Bios can be updated by the
Vendors. times\_at\_fff\_untracked should also be fixed.  Not sure
about vendor\_type.  If their "DBA" name is different from their
vendor\_business\_name then, that can go in the bio field.

The logo submission has yet to be determined, but I figure it's
going to be handled along the same lines as the picture that we
use for Presenters.

The bulk of the changes are in 41\_VendorInfo.sql
\subsubsection{id}
\label{sec-2-1-1}
\begin{itemize}
\item {\bfseries\sffamily DONE} CongoDump.badgeid
\label{sec-2-1-1-1}
\end{itemize}

\subsubsection{vendor\_address}
\label{sec-2-1-2}
\begin{itemize}
\item {\bfseries\sffamily DONE} CongoDump.postaddress1 and CongoDump.postaddress2
\label{sec-2-1-2-1}
\end{itemize}

\subsubsection{vendor\_business\_name}
\label{sec-2-1-3}
\begin{itemize}
\item {\bfseries\sffamily DONE} CongoDump.badgename, Bios - badgeid=id, biotype=name
\label{sec-2-1-3-1}
\end{itemize}

\subsubsection{vendor\_city}
\label{sec-2-1-4}
\begin{itemize}
\item {\bfseries\sffamily DONE} CongoDump.postcity
\label{sec-2-1-4-1}
\end{itemize}

\subsubsection{vendor\_contact\_email}
\label{sec-2-1-5}
\begin{itemize}
\item {\bfseries\sffamily DONE} CongoDump.email
\label{sec-2-1-5-1}
\end{itemize}

\subsubsection{vendor\_contact\_name}
\label{sec-2-1-6}
\begin{itemize}
\item {\bfseries\sffamily DONE} CongoDump.firstname, CongoDump.lastname
\label{sec-2-1-6-1}
\end{itemize}

\subsubsection{vendor\_contact\_phone}
\label{sec-2-1-7}
\begin{itemize}
\item {\bfseries\sffamily DONE} CongoDump.phone
\label{sec-2-1-7-1}
\end{itemize}

\subsubsection{vendor\_country}
\label{sec-2-1-8}
\begin{itemize}
\item {\bfseries\sffamily DONE} CongoDump.postcountry
\label{sec-2-1-8-1}
\end{itemize}

\subsubsection{vendor\_description}
\label{sec-2-1-9}
\begin{itemize}
\item {\bfseries\sffamily DONE} Bios - badgeid=id,, biotype=bio
\label{sec-2-1-9-1}
\end{itemize}

\subsubsection{vendor\_state}
\label{sec-2-1-10}
\begin{itemize}
\item {\bfseries\sffamily DONE} CongoDump.poststate
\label{sec-2-1-10-1}
\end{itemize}

\subsubsection{vendor\_website}
\label{sec-2-1-11}
\begin{itemize}
\item {\bfseries\sffamily DONE} Bios - badgeid=id, biotype=url
\label{sec-2-1-11-1}
\end{itemize}

\subsubsection{vendor\_twitter}
\label{sec-2-1-12}
\begin{itemize}
\item {\bfseries\sffamily DONE} Bios - badgeid=id, biotype=twitter
\label{sec-2-1-12-1}
\end{itemize}

\subsubsection{vendor\_facebook}
\label{sec-2-1-13}
\begin{itemize}
\item {\bfseries\sffamily DONE} Bios - badgeid=id, biotype=facebook
\label{sec-2-1-13-1}
\end{itemize}

\subsubsection{vendor\_fetlife}
\label{sec-2-1-14}
\begin{itemize}
\item {\bfseries\sffamily DONE} Bios - badgeid=id, biotype=fetlife
\label{sec-2-1-14-1}
\end{itemize}

\subsubsection{vendor\_DBA}
\label{sec-2-1-15}
\begin{itemize}
\item {\bfseries\sffamily DONE} Bios - badgeid=id, biotype=dba
\label{sec-2-1-15-1}
\end{itemize}

\subsubsection{vendor\_zipcode}
\label{sec-2-1-16}
\begin{itemize}
\item {\bfseries\sffamily DONE} CongoDump.postzip
\label{sec-2-1-16-1}
\end{itemize}

\subsubsection{vendor\_type}
\label{sec-2-1-17}
\begin{itemize}
\item New? possibly a short list of possible with a mapping to the
vendor?
\item Might also include "community table" as a type?  If this isn't a
short list of pickable things, we might want to make another
table field "vendor/community table" with allowable "V" or "C"
as part of the enum.
\item Vendor\_type should be table, with a selectable (not just pull-down)
list, and set by them.
\end{itemize}
\begin{itemize}
\item {\bfseries\sffamily DONE} Needs to be created/mapped based on other Types descriptions
\label{sec-2-1-17-1}
\begin{itemize}
\item VendorTypes table vendortypeid, vendortypename, vendortypedesc
\item VendorIs table badgeid, vendortypeid
\end{itemize}
\end{itemize}

\subsubsection{times\_at\_fff\_untracked}
\label{sec-2-1-18}
The number of times they vended at the fff that we currently don't
have in the database, so that the generated number below can be
(more) accurately generated.  As we add more instances (as I find
them) this should probably get adjusted, so it's not artificially
inflated. 
\begin{itemize}
\item {\bfseries\sffamily DONE} VendorTimesAdj created as a standalone table vaguely similar to Interested
\label{sec-2-1-18-1}
badgeid, vendortimesadj
\end{itemize}

\subsection{Annual}
\label{sec-2-2}
Most of these have to either be created, or co-opted, since the way
that it was being done before is too baroque and complicated.  Easy
enough to simply make new tables, and reuse little bits, instead
of trying to force everything into the other work-flow, as we did on
an "emergency" and "emergent" basis, originally.
\subsubsection{this event's check}
\label{sec-2-2-1}
\begin{itemize}
\item {\bfseries\sffamily DONE} combination of badgeid and conid to avoid multiple entries for an event.
\label{sec-2-2-1-1}
\end{itemize}

\subsubsection{id - varchar(15)}
\label{sec-2-2-2}
linked to Participants.badgeid
\begin{itemize}
\item {\bfseries\sffamily DONE} VendorAnnualInfo.badgeid
\label{sec-2-2-2-1}
\end{itemize}

\subsubsection{conid - int(11)}
\label{sec-2-2-3}
linked to ConInfo.conid
\begin{itemize}
\item {\bfseries\sffamily DONE} VendorAnnualInfo.conid
\label{sec-2-2-3-1}
\end{itemize}

\subsubsection{created - timestamp}
\label{sec-2-2-4}
\begin{itemize}
\item when applied I think?
\item CORRECT
\end{itemize}
\begin{itemize}
\item {\bfseries\sffamily DONE} VendorAnnualInfo.vendorwhenapplied
\label{sec-2-2-4-1}
\end{itemize}

\subsubsection{digital\_advertising - complex}
\label{sec-2-2-5}
\begin{itemize}
\item y/n or \$ or type?
\item do they want it? or how much?
\item JUST A Y/N
\item Actually a drop-down / select (not a mulit-select)
\item three tables
\end{itemize}
\begin{itemize}
\item {\bfseries\sffamily DONE} a list of the possible default digital advertizing types (BaseDigitalAd)
\label{sec-2-2-5-1}
\begin{itemize}
\item basedigitaladid, basedigitaladname, basedigitaladdesc
\end{itemize}

\item {\bfseries\sffamily DONE} a list of digital advertizing for this event, the prices, and the conid (DigitalAd)
\label{sec-2-2-5-2}
\begin{itemize}
\item digitaladid, basedigitaladid, conid, digitaladprice, display\_order
\end{itemize}

\item {\bfseries\sffamily DONE} a mapping of VendorAnnualInfo.vendoranualid (VendorHasDigitalAd)
\label{sec-2-2-5-3}
\begin{itemize}
\item badgeid, digitaladid, digitaladcount
\end{itemize}
\end{itemize}

\subsubsection{fff\_sponsorship - complex}
\label{sec-2-2-6}
\begin{itemize}
\item y/n or \$ or level?
\item will they sponsor us?
\item JUST A Y/N
\item Actually as a level.
\item three tables
\end{itemize}
\begin{itemize}
\item {\bfseries\sffamily DONE} a list of the possible default digital advertizing types (BaseSponsorLevel)
\label{sec-2-2-6-1}
\begin{itemize}
\item basesponsorlevelid, basesponsorlevelname, basesponsorleveldesc
\end{itemize}

\item {\bfseries\sffamily DONE} a list of digital advertizing for this event, the prices, and the conid (SponsorLevel)
\label{sec-2-2-6-2}
\begin{itemize}
\item sponsorlevelid, basesponsorlevelid, conid, sponsorprice, display\_order
\end{itemize}

\item {\bfseries\sffamily DONE} a mapping of VendorAnnualInfo.vendoranualid (VendorHasSponsorLevel)
\label{sec-2-2-6-3}
\begin{itemize}
\item badgeid, sponsorlevelid, sponsorlevelcount
\end{itemize}
\end{itemize}

\subsubsection{print\_advertising - complex}
\label{sec-2-2-7}
\begin{itemize}
\item y/n or \$ or type?
\item do they want it?
\item JUST A Y/N
\item Multi-select (drop down) as a possiblity.
\item Bag stuffer
\item three tables
\end{itemize}
\begin{itemize}
\item {\bfseries\sffamily DONE} a list of the possible default digital advertizing types (BaseDigitalAd)
\label{sec-2-2-7-1}
\begin{itemize}
\item baseprintadid, baseprintadname, baseprintaddesc
\end{itemize}

\item {\bfseries\sffamily DONE} a list of digital advertizing for this event, the prices, and the conid (DigitalAd)
\label{sec-2-2-7-2}
\begin{itemize}
\item printadid, baseprintadid, conid, printadprice, display\_order
\end{itemize}

\item {\bfseries\sffamily DONE} a mapping of VendorAnnualInfo.vendoranualid (VendorHasDigitalAd)
\label{sec-2-2-7-3}
\begin{itemize}
\item badgeid, printadid, printadcount
\end{itemize}
\end{itemize}

\subsubsection{updated - timestamp}
\label{sec-2-2-8}
\begin{itemize}
\item most recently updated, I believe
\item CORRECT
\end{itemize}
\begin{itemize}
\item {\bfseries\sffamily DONE} VendorAnnualInfo.vendorupdated
\label{sec-2-2-8-1}
\end{itemize}

\subsubsection{vendor\_acknowledgement - varchar(50) (signature)}
\label{sec-2-2-9}
\begin{itemize}
\item Vendor acknowledges \ldots{} something?
\item VENDOR ACKNOWLEDGES THEY AGREE TO OUR TERMS AND CONDITIONS -
THIS SHOULD BE AGREED TO ANNUALLY
\item Possible second page, with what the agreement is.
\end{itemize}
\begin{itemize}
\item {\bfseries\sffamily DONE} VendorAnnualInfo.vendoracknowledgement
\label{sec-2-2-9-1}
\end{itemize}

\subsubsection{vendor\_additional\_notes / additional\_information - text}
\label{sec-2-2-10}
\begin{itemize}
\item this might also want to be perennial as well, perhaps?
\item ANNUAL ONLY; THIS IS REGARDING REQUESTS (EX:  FUR ALLERGY, WALL,
ETC)
\end{itemize}
\begin{itemize}
\item {\bfseries\sffamily DONE} VendorAnnualInfo.vendornotes
\label{sec-2-2-10-1}
\end{itemize}

\subsubsection{vendor\_amenities\_foo - ?? (see also VendorFeatures)}
\label{sec-2-2-11}
\begin{itemize}
\item A variety of possible fields?  How is this  used?
\begin{itemize}
\item 6ft\_table
\item corner\_endcap
\item extra\_badges
\item number\_of\_chairs
\item shared\_electrical 8ft\_tables
\end{itemize}
\item RIGHT NOW IT'S SPLIT INTO MULTIPLE LINES WITH MULTI OPTIONS.  \#
OF TABLES, \# OF CHAIRS, CORNER/END CAP, ELECTRICAL, EXTRA
BADGES
\item three tables
\end{itemize}
\begin{itemize}
\item {\bfseries\sffamily TODO} need to migrate old information somehow, as well as build the three tables
\label{sec-2-2-11-1}
\item {\bfseries\sffamily DONE} BaseVendorFeature
\label{sec-2-2-11-2}
basevendorfeatureid, basevendorfeaturename, basevendorfeaturedesc

\item {\bfseries\sffamily DONE} VendorFeature based on VendorFeatures
\label{sec-2-2-11-3}
vendorfeatureid, basevendorfeatureid, conid, vendorfeatureprice,
display\_order

\item {\bfseries\sffamily DONE} create VendorHasFeature
\label{sec-2-2-11-4}
badgeid, vendorfeature, vendorfeaturecount
\end{itemize}

\subsubsection{vendor\_contract - varchar(5)}
\label{sec-2-2-12}
\begin{itemize}
\item How is this used?
\item THIS IS INITIALS TO STATE THEY READ AND AGREE TO OUR TERMS ALREADY ABOVE
\item Probably not used, see above vendoracknowledgement
\end{itemize}
\begin{itemize}
\item {\bfseries\sffamily DONE} see VendorAnnualInfo.vendoracknowledgement
\label{sec-2-2-12-1}
\end{itemize}

\subsubsection{vendor\_invoiced - Y/N (enum)}
\label{sec-2-2-13}
\begin{itemize}
\item Were they invoiced?
\item YUP
\item more complex, town invoice, and then us invoice.  Details probably
involved as to what the invoice is for.
\end{itemize}
\begin{itemize}
\item {\bfseries\sffamily DONE} part of Vendor Status
\label{sec-2-2-13-1}
\end{itemize}

\subsubsection{vendor\_location - ??}
\label{sec-2-2-14}
\begin{itemize}
\item Where we put them.
\item YUP
\item This might want to be slightly more complex, building, room, and
booth-number
\end{itemize}
\begin{itemize}
\item {\bfseries\sffamily TODO} VendorLocation table?
\label{sec-2-2-14-1}
\begin{itemize}
\item badgeid, conid, building, room (tied to rooms?), booth number
\item Might just do as a free form field for this year, and make it
more complex for the future.
\end{itemize}
\end{itemize}

\subsubsection{vendor\_payment\_adjustment - decimal(8,2)}
\label{sec-2-2-15}
\begin{itemize}
\item Adjustments to the bill
\item YUP
\item chance for adjustments (like shitty corner)
\end{itemize}
\begin{itemize}
\item {\bfseries\sffamily DONE} VendorAnnualInfo.vendorpayadj
\label{sec-2-2-15-1}
\end{itemize}

\subsubsection{vendor\_payment\_amount - decimal(8,2)}
\label{sec-2-2-16}
\begin{itemize}
\item amount they paid, should match invoiced amounts
\item YUP
\item differentiated by the above adjustment
\end{itemize}
\begin{itemize}
\item {\bfseries\sffamily TODO} not sure how it's generated?
\label{sec-2-2-16-1}
\item {\bfseries\sffamily DONE} VendorAnnualInfo.vendorpaid
\label{sec-2-2-16-2}
\end{itemize}

\subsubsection{vendor\_payment\_received - Y/N (enum)}
\label{sec-2-2-17}
\begin{itemize}
\item check box?
\item YES
\item default "N" or maybe blank?
\item Possibly No (blank?), Warwick, and In-full as field entries.
\end{itemize}
\begin{itemize}
\item {\bfseries\sffamily DONE} part of Vendor Status
\label{sec-2-2-17-1}
\end{itemize}

\subsubsection{vendor\_preferred\_space - ?? (see also VendorSpaces)}
\label{sec-2-2-18}
\begin{itemize}
\item Where they would like to be put.
\item NO - THIS IS BOOTH SIZE (SINGLE, DOUBLE, ETC)
\item What is requested - several options, this is just your
preference, no guarentees.  (select at most 3)
\item four tables
\end{itemize}
\begin{itemize}
\item {\bfseries\sffamily DONE} BaseVendorSpace based on VendorSpaces
\label{sec-2-2-18-1}
basevendorspaceid, basevendorspacename, basevendorspacedesc

\item {\bfseries\sffamily DONE} VendorSpace based on VendorSpaces
\label{sec-2-2-18-2}
\begin{itemize}
\item This event's options, based on VendorSpaces
\item vendorspaceid, basevendorspaceid, conid, vendorspaceprice,
display\_order
\end{itemize}

\item {\bfseries\sffamily DONE} VendorPrefSpace
\label{sec-2-2-18-3}
badgeid, vendorspace, vendorspacerank (enum 1st, 2nd, 3rd)

\item {\bfseries\sffamily DONE} VendorHasSpace
\label{sec-2-2-18-4}
badgeid, vendorspace, vendorspacecount
\end{itemize}

\subsubsection{vendor\_loadin\_self\_carry - Y/N (enum)}
\label{sec-2-2-19}
\begin{itemize}
\item will they?
\item This should be Y/n, right?
\item YES
\end{itemize}
\begin{itemize}
\item {\bfseries\sffamily DONE} VendorAnnualInfo.vendorselfcarry
\label{sec-2-2-19-1}
\end{itemize}

\subsubsection{vendor\_sponsorship\_package  - ??}
\label{sec-2-2-20}
\begin{itemize}
\item what package they wanted?  They actually bought?
\item THIS IS THE Y/N ABOVE
\item if this is redundant it should be removed.  If it is the same as
what is in missing, below, then \ldots{} perhaps not so missing after
all?
\item Redundant with the first one.
\end{itemize}
\begin{itemize}
\item {\bfseries\sffamily DONE} with the above.  Make sure the data is migrated appropriately.
\label{sec-2-2-20-1}
\end{itemize}

\subsubsection{vendor\_status / status - Denied/Approved/Accepted/Approved/Paid (sub-table)}
\label{sec-2-2-21}
\begin{itemize}
\item STATUS
\item DENIED, APPROVED, ACCEPTED, APPROVED AND PAID
\item Approved is in twice?  Should it be 4, Denied, Approved,
Accepted, Paid?  I'd also add "Proposed" or the like?
\item Proposed, Applied, NELA Approval, Warwick Invoiced, Pending
Warwick Approval, Warwick Approval, NELA Invoiced, Paid,
Accepted, NELA Denied, Warwick Denied, Banned.
\end{itemize}
\begin{itemize}
\item {\bfseries\sffamily DONE} probably based on Interested/InterestedTypes
\label{sec-2-2-21-1}
VendorStatusTypes - vendortypeid, display order, vendortypename,
vendortypedesc -- populated.
VendorStatus - conid, badgeid, vendorstatustypeid
\end{itemize}

\subsubsection{rejected\_reason - text}
\label{sec-2-2-22}
\begin{itemize}
\item KEEP, WOULD BE SUBJECT TO CHANGE UNLESS VENDOR IS BANNER
\end{itemize}
\begin{itemize}
\item {\bfseries\sffamily DONE} VendorAnnualInfo.vendordenyreason
\label{sec-2-2-22-1}
\end{itemize}

\subsubsection{Tracking fields similar to participant tracking on what was done.}
\label{sec-2-2-23}
\begin{itemize}
\item {\bfseries\sffamily DONE} based on SessionEditHistory/SessionEditCodes/NotesOnParticipants
\label{sec-2-2-23-1}
\begin{itemize}
\item VendorEditHistory: vbadgeid, conid, badgeid, vendoreditcode,
vendorstatustypeid, vendorchangets, vendordelta
\item VendorEdit Codes: vendoreditcode. display\_order, vendoreditname
\end{itemize}
\end{itemize}

\subsection{Missing}
\label{sec-2-3}
These are values that I think we need, that don't seem to actually
exist anywhere else, and should probably be at least tracked, if
not having several layers to them, so that they can be figured into
things like costs, and historical data.
\subsubsection{vendor\_actual\_space}
\label{sec-2-3-1}
\begin{itemize}
\item single, double, room, etc. probably based on the same
VendorSpaces/VendorHasVendorSpace etc.
\item Us modifyable, not them modifiable.
\end{itemize}
\begin{itemize}
\item {\bfseries\sffamily DONE} see above under VendorHasSpace
\label{sec-2-3-1-1}
\end{itemize}

\subsection{Generated}
\label{sec-2-4}
These values would not actually be stored anywhere, but generated
for various purposes, from other information collected in the
database, rather than simple values.  The Vendor Invoice Amount
should also go along with the Vendor Invoice, which lists all the
things they are paying for, and interfaces with our credit-card
system.
\subsubsection{vendor\_invoice\_amount - decimal(13,4)}
\label{sec-2-4-1}
\begin{itemize}
\item Money
\item YUP
\item possibly a geneerated field, as opposed to a stored one, based
on the varieties chosen in vendor\_amenites,
vendor\_payment\_adjustment, vendor\_actual\_space and others?
\end{itemize}

\subsubsection{times\_at\_fff}
\label{sec-2-4-2}
\begin{itemize}
\item currently a freeform field, should probably be generated, with a
special number that adds in, for (currently) untracked events.
\item SOUNDS GOOD
\item Base figure above in perennial
\end{itemize}

\subsection{Structure}
\label{sec-2-5}
These are the proposed new tables or modifications to existing
tables that the data will be collected in.  Mostly interesting for
building purposes, but also for those curious on how things operate
under the hood, because, I like sharing that way, as opposed to
keeping things secret and cryptic.  Yes, I know it's called "code"
because we didn't want you to know what it did, but \ldots{} 
\subsubsection{Vendor Constants}
\label{sec-2-5-1}
\begin{itemize}
\item Most of these (as listed) are already in the CongoDump table.
\item Two need their own table:
\begin{itemize}
\item vendor\_type
possibly a short list of possible with a mapping to the
vendor, so might be another table
\item times\_at\_fff\_untracked - int(3)
\end{itemize}
\end{itemize}

\subsubsection{Vendor Yearly (by Event) Variables}
\label{sec-2-5-2}
\begin{itemize}
\item Most just are, as specified in this table
\item Some might be cross-references to the other tables below, or be
in the other tables entirely.
\end{itemize}

\subsubsection{Vendor Features}
\label{sec-2-5-3}
Probably just a tweak to this table, and might need to be a
"standard set" and a "this event set" like BaseFeatures and
Features does as well as BaseServices and Services does and a
repurposing of "SessionHasVendorFeature" to
"VendorHasVendorFeature"

\subsubsection{Vendor Spaces}
\label{sec-2-5-4}
Probably just a tweak to this table, and might need to be a
"standard set" and a "this event set" like BaseFeatures and
Features does as well as BaseServices and Services does and a
repurposing of "SessionHasVendorSpace" to "VendorHasVendorSpace"

\subsubsection{Vendor Location}
\label{sec-2-5-5}
Might be it's own table, with conid (linked), badgeid (linked),
building, floor, room (possibly linked), and booth\_number, and
then code around it so blanks just don't get reported (aka, if
they are a room vendor, their booth\_number might not be listed,
but if there are several vendors in a room, it might be.)  This
might also allow ties to maps, and possibly the redraw of maps.

\subsubsection{Vendor Status}
\label{sec-2-5-6}
Might want to be it's own table as a mapping, so searches, and
display\_order could be set.

\subsubsection{Vendor Sponsorship Package/vendor\_sponsor\_level}
\label{sec-2-5-7}
These three would need to be determined.

\subsubsection{digital\_advertising\_level}
\label{sec-2-5-8}
These three would need to be determined.

\subsubsection{print\_advertising\_level}
\label{sec-2-5-9}
These three would need to be determined.

\subsubsection{More structure in the 41\_VendorInfo.sql file}
\label{sec-2-5-10}

\subsection{Still Unknown}
\label{sec-2-6}
These are the variables that exist already in the other usage
space, that I still don't have a handle on.  More explination would
be very useful.
\subsubsection{created\_by - number}
\label{sec-2-6-1}
\begin{itemize}
\item Not sure what this does?
\item I THINK IT"S VENDORS NAME VS. I UPDATED, ETC
\item So \ldots{} how should it be expressed/tracked?
\item Probably covered by VendorAnnualInfo.vendorupdated tracking (above)
\end{itemize}

\subsubsection{ordering\_count - number}
\label{sec-2-6-2}
\begin{itemize}
\item Another random number?
\item YES
\item Should it be kept?  Removed, what is it's state?  What is it's
purpose?
\item timestamp on entry rather than something else, kept in
timestamp, otherwise \ldots{} who knows.
\item Probably covered by VendorAnnualInfo.vendorwhenapplied
\end{itemize}

\section{Front End}
\label{sec-3}
\subsection{Available Pages}
\label{sec-3-1}
This should be the comprehensive list of all the pages that either
the vendors/community table folk need to interact with or those who
manage them (the Div Head) needs to interact with.
\subsubsection{Returning Vendor sign-in}
\label{sec-3-1-1}
By number or email address.
\begin{itemize}
\item Update page (subset of new apply page)
\item apply page (subset of new apply page)
\end{itemize}
\begin{itemize}
\item {\bfseries\sffamily VENDORQA} needs to be vetted.
\label{sec-3-1-1-1}
\begin{itemize}
\item Possibly add the SuperVendor update here.
\end{itemize}
\end{itemize}

\subsubsection{Returning Community Table sign-in}
\label{sec-3-1-2}
By number or email address.  Might simply be switched on one of
the values in the vending type or on it's own value, and just have
one place to go?  Since we are collecting pretty much the same
information for them both?  Or should I be separating them out?
\begin{itemize}
\item not for this year, but community table a separate page (at least
\end{itemize}
for winter, since they don't have to apply)
\begin{itemize}
\item BIG NOTE: any money exchanged makes you a vendor, Be aware
\end{itemize}
\begin{itemize}
\item {\bfseries\sffamily TODO} Currently simply a subset of vendor information.
\label{sec-3-1-2-1}
\end{itemize}

\subsubsection{New Vendor proposal}
\label{sec-3-1-3}
With a catch to see if they are already in our database, using
their email address
\begin{itemize}
\item {\bfseries\sffamily VENDORQA} final vetting by the vendor folks.
\label{sec-3-1-3-1}
\end{itemize}

\subsubsection{New Community Table proposal}
\label{sec-3-1-4}
With a catch to see if they are already in our database, using
their email address
\begin{itemize}
\item not for this year, but community table a separate page (at least
\end{itemize}
for winter, since they don't have to apply)
\begin{itemize}
\item {\bfseries\sffamily TODO} Currently simply a subset of vendor information.
\label{sec-3-1-4-1}
\end{itemize}

\subsubsection{Returning Vendor/Community Table informational gathering}
\label{sec-3-1-5}
This can be the same, although there might, again, be a slight
difference bases on vendor/community table differences.  Also, we
might want to present last? event's information as defaults, so
they have to do less typing. Dealing with changes once the state
has changed to "invoice generated" might require Vendor Div Head
intervention.
\begin{itemize}
\item Steps to do this.  This will be several "pages" as the Welcome
\end{itemize}
Page, depending on status.
\begin{itemize}
\item {\bfseries\sffamily VENDORQA} Proposed (1) / Applied (2)
\label{sec-3-1-5-1}
Either Proposed (by someone else) or Applied (basically filling
out the form we are giving, or (which isn't set up yet) a
returning Vendor, filling out a shorter form, but will all the
appropriate event information on it.
\begin{itemize}
\item Waiting on Vendor QA of the words, and the contract.
\end{itemize}

\item {\bfseries\sffamily VENDORQA} NELA Approved (3)
\label{sec-3-1-5-2}
Once a set of the proposals are in, the jury decides if their
proposal is accepted and acceptable, the NELA Approved button for
them is clicked, email is sent to them letting them know that
they are a NELA approved vendor, and they should log in (with
their log in information) to go to the next step.  They see the
second page when they log in, which has the information about
what Warwick is doing differently this year, and a link to pull
down the PDF to be filled in, and a way to submit the filled-out
PDF.
\begin{itemize}
\item Not sure if we should be auto-moving on.
\end{itemize}

\item {\bfseries\sffamily VENDORQA} Warwick Invoced (4)
\label{sec-3-1-5-3}
Once the PDF has been submitted and/or when the next button has
been clicked, they are then presented with an invoice for the
Warwick fees.
\begin{itemize}
\item Still needs invoice testing, need a credit card for that.
\end{itemize}

\item {\bfseries\sffamily VENDORQA} Pending Warwick Approval (5)
\label{sec-3-1-5-4}
Once the fees have been paid and/or when the next button has been
clicked, they are given the message, when they log in, that they
are still pending Warwick's approval.

\item {\bfseries\sffamily VENDORQA} Warwick Approval (6)
\label{sec-3-1-5-5}
Once we get Warwick's approval, the button is clicked and they
will see the message, something to the effect of "Warwick has
approved, this is your current order, do you want to change
anything before your invoicing (this is where the various
digital, print, and sponsorship options, as well as the amenities
can be changed, as well).

\item {\bfseries\sffamily VENDORQA} NELA Invoiced (7)
\label{sec-3-1-5-6}
Once they indicate that their invoice is fine, the invoice is
presented.  They then can pay it.
\begin{itemize}
\item Still needs testing, need a credit card for that.
\item Although the testing might be covered by Warwick Invoiced (4)
above.
\end{itemize}

\item {\bfseries\sffamily VENDORQA} Paid (8)
\label{sec-3-1-5-7}
This is the state they will be in once the invoice is paid.  At
this point, it can flow directly to the next step, but I have
something (somewhat confusing in my notes) to the effect that
there wants to be a check \ldots{} sent to them? or sent to us? or
possibly that we are waiting on a check, or something else, that
means that they are waiting in this state.

\item {\bfseries\sffamily VENDORQA} Accepted (9)
\label{sec-3-1-5-8}
Once this button is clicked, then everything comes up roses, they
are on our websites, and visible, and all is good in the world.

\item {\bfseries\sffamily VENDORQA} NELA Denied (10)
\label{sec-3-1-5-9}
For those vendors who just didn't make the cut (this event?)
usually would happen after (or instead of) the NELA Approved (3)
state.

\item {\bfseries\sffamily VENDORQA} Warwick Denied (11)
\label{sec-3-1-5-10}
For those vendors who didn't make Warwick's cut this event (and
can they apply again, subsequent events?  Are we going to track
the reason Warwick rejected them?  Will we even know the reason?)
usually would happen after (or instead of) the Warwick Approval
state.

\item {\bfseries\sffamily VENDORQA} Banned (12)
\label{sec-3-1-5-11}
So we track those who are banned, so if they try to apply, we
know.  Probably clunky, and \ldots{} not sure at what point in the
process they should be informed.  I'm thinking at about the NELA
Denied, right after (or instead of) the NELA Approved state as
well.
\end{itemize}

\subsubsection{Returning Vendor informational update}
\label{sec-3-1-6}
This might not be necessary, based on the above page keeping what
is set for this event as set.
\begin{itemize}
\item {\bfseries\sffamily DONE} included in above Returning set of pages.
\label{sec-3-1-6-1}
\end{itemize}

\subsubsection{Returning Vendor/Community Table Bio update}
\label{sec-3-1-7}
Page that they can, at will update their web or book information,
along with all their other invormation.
\begin{itemize}
\item {\bfseries\sffamily VENDORQA} similar to the above set of pages, but allowed always.
\label{sec-3-1-7-1}
\end{itemize}

\subsubsection{Invoice}
\label{sec-3-1-8}
Not available until after both they are accepted (by Vendor
Div Head), and they hit the "generate invoice" button, which
probably should lock changes.  Once the invoice is generated, it
should be fixed, and also payable.
This will (of course) be a comprehensive list of everything they
agreed to.  The invoice page should continue to be available after
they paid, so there is a referece point to the services agreed
to/paid for.
\begin{itemize}
\item Yes, included in the above
\end{itemize}
\begin{itemize}
\item {\bfseries\sffamily DONE} included in above Returning set of pages.
\label{sec-3-1-8-1}
\end{itemize}

\subsubsection{Ad Copy Submissions}
\label{sec-3-1-9}
For those who have agreed to either digital or print ads, this is
a place for them to upload/submit them.
\begin{itemize}
\item Probably offline, going directly to the AD folk.
\end{itemize}
\begin{itemize}
\item {\bfseries\sffamily TODO} (perhaps, at some point?)
\label{sec-3-1-9-1}
\end{itemize}

\subsubsection{Vendor Jury}
\label{sec-3-1-10}
The page that all the proposed vendors, with certain bits of
information (to be determined, since I've not seen that page yet
in our current system) will be on for the Vendor Div Head to move
vendors to either "Accepted" or "Denied".  Probably includes some
vendor notes and the like, including "Denied Reasons" Also
probably where any price adjustments happen to the invoice.  Maybe
a grouping by Vendor Type, and leaving the ones already moved on
from this in a different colour, so that those already accepted,
and their type, will be shown for balance reasons.
\begin{itemize}
\item {\bfseries\sffamily VENDORQA} in pieces
\label{sec-3-1-10-1}
\begin{itemize}
\item first report - which allows for changing the Vendor State is
done.
\item set of tunable reports
\item voting page, much like other voting pages?
\end{itemize}
\end{itemize}

\subsubsection{Vendor Placement}
\label{sec-3-1-11}
After the invoice is paid? Or perhaps slightly different, if there
is those that have invoice exceptions allowing for the placement
of vendors.  Probably should have the invoice amount paid, and
invoice amount expected on this page, to verify that they are,
indeed, all paid up.  Also the vendor type should show up here, as
well, so that similars can either be grouped together, or spread
apart, as per the decision of the Div Head. May also contain
previous placements, or links to previous maps or something useful
along those lines.  Including the current svg of the map.
\begin{itemize}
\item {\bfseries\sffamily TODO} 
\label{sec-3-1-11-1}
\end{itemize}

\subsubsection{Administer Vendor Options}
\label{sec-3-1-12}
Page that has the Amenities and Spaces for this event, pulled from
the standard pool.  To be updated by the Vendor Coordinator on an
event by event basis.
\begin{itemize}
\item {\bfseries\sffamily VENDORQA} 
\label{sec-3-1-12-1}
\end{itemize}

\subsubsection{Adminsiter Ad/Sponsor Options}
\label{sec-3-1-13}
Page that has the Sponsor Levels, Digital Ads and Print Ads for
this event, pulled from the standard pool.  To be updated either
by the Vendor Div Head or the Publications Div Head on an event by
event basis.
\begin{itemize}
\item {\bfseries\sffamily VENDORQA} 
\label{sec-3-1-13-1}
\begin{itemize}
\item Page is done, need the base information so that the event
information can be selected.
\end{itemize}
\end{itemize}

\subsubsection{Administer Vendor}
\label{sec-3-1-14}
Where any changes are made to the vendor information (both that
they can change and that they can't) for creation/update by Vendor
Div Head.
\begin{itemize}
\item {\bfseries\sffamily VENDORQA} 
\label{sec-3-1-14-1}
\begin{itemize}
\item Still a little wonky, brings back to a less-useful page, but
still works.  Can select by email address or by "pubsname"
which is the business name.
\end{itemize}
\end{itemize}

\subsubsection{Vendor Welcome Letter}
\label{sec-3-1-15}
Generate the (physical) welcome letter to the vendors
\begin{itemize}
\item {\bfseries\sffamily TODO} 
\label{sec-3-1-15-1}
\end{itemize}

\subsubsection{Vendor Email}
\label{sec-3-1-16}
Generate (and send) various all-vendor communications
\begin{itemize}
\item {\bfseries\sffamily TODO} 
\label{sec-3-1-16-1}
\end{itemize}

\subsubsection{Vendor Badges}
\label{sec-3-1-17}
Generate the Vendor Badges for printing purposes.
\begin{itemize}
\item No!
\end{itemize}
\begin{itemize}
\item {\bfseries\sffamily TODO} even if we won't use them.
\label{sec-3-1-17-1}
\end{itemize}

\subsubsection{Vendor Tents}
\label{sec-3-1-18}
Generate all the vendor "tents" to put on their tables, to mark
their place.
\begin{itemize}
\item Business Name (and DBA?) and Booth \#
\end{itemize}
\begin{itemize}
\item {\bfseries\sffamily TODO} 
\label{sec-3-1-18-1}
\end{itemize}

\subsubsection{Check-in/Check-out lists}
\label{sec-3-1-19}
All of the vendors expected, so they can be ticked off the list
when they get there, and, so that when they return whatever they
have to return at the end of the night (tax forms, possibly?) can
be ticked off again.
\begin{itemize}
\item Dock forms (room vendors vs otherwise) Each one Business Name
and Booth number.  (several copies)
\item Master List for check in.  Second for Check out. RI vendor
marked specially.
\end{itemize}
\begin{itemize}
\item {\bfseries\sffamily TODO} 
\label{sec-3-1-19-1}
\end{itemize}

\subsection{Flow}
\label{sec-3-2}
\subsubsection{Set Amenities}
\label{sec-3-2-1}
Go to the VendorSetupSpaceFeature.php page, and set the
appropriate amenities/prices/et al for this event.

\subsubsection{Set Spaces}
\label{sec-3-2-2}
Go to the VendorSetupSpaceFeature.php page, and set the
appropriate spaces/prices/et al for this event.

\subsubsection{Set Sponsorship levels}
\label{sec-3-2-3}
Go to the PubsSetupAds.php page, and set the
appropriate sponsor levels available/prices/et al for this event.

\subsubsection{Set Digital Ad possiblities}
\label{sec-3-2-4}
Go to the PubsSetupAds.php page, and set the
appropriate digital ads available/prices/et al for this event.

\subsubsection{Set Printed Ad possibilities}
\label{sec-3-2-5}
Go to the PubsSetupAds.php page, and set the
appropriate print ads available/prices/et al for this event.

\subsubsection{Task List}
\label{sec-3-2-6}
Task List element with the call for vendors is created, with the
appropriate link.

\subsubsection{Task List}
\label{sec-3-2-7}
Task List element with the call for community tables is created,
with the appropriate link.

\subsubsection{Email}
\label{sec-3-2-8}
Possible email from Vendor Div Head goes out reminding
vendors/community tables when the call will open and close.

\subsubsection{Phase Change}
\label{sec-3-2-9}
At the appropriate time, the Con Chair or Zambia bitch open the
call for vendors.

\subsubsection{Email}
\label{sec-3-2-10}
Vendor Div Head generates and sends email letting the
vendors/community tables know that the call is open.

\subsubsection{Vendors/Community Table folk apply}
\label{sec-3-2-11}
They use either the returning or the new pages, select the
appropriate features, sponsorships, and ads that they want, and
make sure everything is as expected.

\subsubsection{(If Rolling Acceptance) Jury}
\label{sec-3-2-12}
If there is a rolling acceptantce of vendors and/or community
tables, while the vendor call/community table call is open
acceptances can roll out and then things can go forward.
Otherwise \ldots{}

\subsubsection{Phase Change}
\label{sec-3-2-13}
At the appropriate time, the Con Chair or Zambia bitch close the
call for vendors.

\subsubsection{Jury}
\label{sec-3-2-14}
The (rest of (if rolling acceptance is happening)) vendors and
community tables get acceptance, which probably is an individual
email from the Div Head, with any specific to them in information
included, and an invitation for them to pay their invoice.

\subsubsection{Invoice}
\label{sec-3-2-15}
Vendors then (hopefully, magically, smoothly) pay their agreed-to
invoices.

\subsubsection{Ad Copy}
\label{sec-3-2-16}
All paid for advertizing copy is generated by the vendors or
community tables is submitted.

\subsubsection{Placement}
\label{sec-3-2-17}
Div Head (and anyone else that can be roped into helping)
schedules the vendors to be put in the appropriate locations on
the day of the event.

\subsubsection{Email}
\label{sec-3-2-18}
Final email goes out, with any instruction reminders and if there
is a manditory vendor meeting, or the like.

\subsubsection{Welcome Letter}
\label{sec-3-2-19}
Welcome lettters generated and printed, possibly stuffed in an
envelope or folder of some sort.
This might include some form of their (electronically) signed
contract, and maybe invoice highlights.  Definitely includes their
placement information, and information on how load-in and load-out
works, etc.

\subsubsection{Badges}
\label{sec-3-2-20}
Badges, the appropriate number for each vendor and community
table, are printed, and possibly stuffed as above.

\subsubsection{Wrist bands}
\label{sec-3-2-21}
Are possibly stuffed as above.

\subsubsection{Check-in/check-out lists printed}
\label{sec-3-2-22}
So they are available on the day-of.

\section{Things to add, from the new:}
\label{sec-4}
\subsection{Logo to be uploaded}
\label{sec-4-1}
\subsection{PDF to be updated}
\label{sec-4-2}

\section{Conclusion}
\label{sec-5}
This should cover all the bits and pieces.  If I missed any, let me
know so this document can be updated, and the system can be tested.
Hopfully all will work, and this can go forth as The New Way.  Until
we change it, again, in 2-3 years \textbf{grin}.
\end{document}