\documentclass[captions=tablesignature]{scrartcl}
\usepackage[utf8]{inputenc}
\usepackage[T1]{fontenc}
\usepackage{fixltx2e}
\usepackage{graphicx}
\usepackage{longtable}
\usepackage{float}
\usepackage{wrapfig}
\usepackage{rotating}
\usepackage[normalem]{ulem}
\usepackage{amsmath}
\usepackage{textcomp}
\usepackage{marvosym}
\usepackage{wasysym}
\usepackage{amssymb}
\usepackage{hyperref}
\tolerance=1000
\usepackage{booktabs}
\usepackage[scaled]{beraserif}
\usepackage[scaled]{berasans}
\usepackage[scaled]{beramono}
\usepackage[usenames,dvipsnames]{color}
\usepackage{fancyhdr}
\usepackage{subfig}
\usepackage{listings}
\lstnewenvironment{common-lispcode}
{\lstset{language={HTML},basicstyle={\ttfamily\footnotesize},frame=single,breaklines=true}}
{}
\usepackage{paralist}
\let\itemize\compactitem
\let\description\compactdesc
\let\enumerate\compactenum
\usepackage[letterpaper,includeheadfoot,top=12.5mm,bottom=25mm,left=19mm,right=19mm]{geometry}
\pagestyle{fancy}
\setcounter{secnumdepth}{2}
\author{Percy, Amanda, Rob, Iya\thanks{NELA.Percy@gmail.com}}
\date{May 2017}
\title{Vending Design Document}
\hypersetup{
  pdfkeywords={Zambia, Documentation, FFF branch},
  pdfsubject={Zambia is a piece of Con Management Software.  This document is a guide to building the Vendor System for the Zambia FFF-branch instance.  This is still a work in progress.},
  pdfcreator={}}
\begin{document}

\maketitle
\pagenumbering{roman}
\thispagestyle{fancy}
\renewcommand{\headrulewidth}{0pt}
\renewcommand{\footrulewidth}{0pt}
\lhead{}
\rhead{}
\chead{}
\lfoot{}
\cfoot{}
\rfoot{}
\begin{abstract}
\vspace{5cm}
{\LARGE{\textbf{Abstract:\\}}}
Zambia is a piece of Con Management Software.  This document is a guide to building the Vendor System for the Zambia FFF-branch instance.  This is still a work in progress.
\end{abstract}
\newpage
\renewcommand{\headrulewidth}{1pt}
\renewcommand{\footrulewidth}{1pt}
\chead{
Vending Design Document
}
\lfoot{
Percy, Amanda, Rob, Iya <NELA.Percy@gmail.com>
}
\rfoot{\thepage}
\setcounter{tocdepth}{2}
\tableofcontents
\listoftables
\listoffigures
\newpage
\pagenumbering{arabic}
\section{General Design}
\label{sec-1}
This is a working guide on how the vending portion of Zambia will be
designed/changed.  Some of it is already in place, some has yet to
be built.  This guide should have all the relevant bits in it so
that all the stake-holders can have their say, and we are working
off of one document.

NOTE: when there is a - or a + next to things, it's because I asked
that as a question, or made a supposition, and it was answered (by
Amanda) so, for now at least the - is my side of the conversation
and the + is her side of the conversation.

\section{Database Design}
\label{sec-2}
\subsection{Perennial}
\label{sec-2-1}
These are the current values that should be migrated to the Zambia
values if they are not already set.  The latest entries will be
considered cannonical.  The ones of these in CongoDump will be more
or less fixed and (currently) take an act of Vendor Coordinator (or
other back-end Zambia person) to update.  

The ones of these in Bios can be updated by the
Vendors. times\_at\_fff\_untracked should also be fixed.  Not sure
about vendor\_type.  If their "DBA" name is different from their
vendor\_business\_name then, that can go in the bio field.

The logo submission has yet to be determined, but I figure it's
going to be handled along the same lines as the picture that we
use for Presenters.
\begin{itemize}
\item id
\label{sec-2-1-1}
CongoDump.badgeid

\item vendor\_address
\label{sec-2-1-2}
CongoDump.postaddress1 and CongoDump.postaddress2

\item vendor\_business\_name
\label{sec-2-1-3}
CongoDump.badgename, Bios - badgeid=id, biotype=name

\item vendor\_city
\label{sec-2-1-4}
CongoDump.postcity

\item vendor\_contact\_email
\label{sec-2-1-5}
CongoDump.email

\item vendor\_contact\_name
\label{sec-2-1-6}
CongoDump.firstname, CongoDump.lastname

\item vendor\_contact\_phone
\label{sec-2-1-7}
CongoDump.phone

\item vendor\_country
\label{sec-2-1-8}
CongoDump.postcountry

\item vendor\_description
\label{sec-2-1-9}
Bios - badgeid=id,, biotype=bio 

\item vendor\_state
\label{sec-2-1-10}
CongoDump.poststate

\item vendor\_website
\label{sec-2-1-11}
Bios - badgeid=id, biotype=url

\item vendor\_twitter
\label{sec-2-1-12}
Bios - badgeid=id, biotype=twitter

\item vendor\_facebook
\label{sec-2-1-13}
Bios - badgeid=id, biotype=facebook

\item vendor\_fetlife
\label{sec-2-1-14}
Bios - badgeid=id, biotype=fetlife

\item vendor\_zipcode
\label{sec-2-1-15}
CongoDump.postzip

\item vendor\_type
\label{sec-2-1-16}
\begin{itemize}
\item New? possibly a short list of possible with a mapping to the
vendor?
\item Might also include "community table" as a type?  If this isn't a
short list of pickable things, we might want to make another
table field "vendor/community table" with allowable "V" or "C"
as part of the enum.
\end{itemize}

\item times\_at\_fff\_untracked
\label{sec-2-1-17}
The number of times they vended at the fff that we currently don't
have in the database, so that the generated number below can be
(more) accurately generated.  As we add more instances (as I find
them) this should probably get adjusted, so it's not artificially
inflated.
\end{itemize}

\subsection{Annual}
\label{sec-2-2}
Most of these have to either be created, or co-opted, since the way
that it was being done before is too baroque and complicated.  Easy
enough to simply make new tables, and resuse little bits, instead
of trying to force everything into the other workflow, as we did on
an "emergency" and "emergant" basis, originally.
\begin{itemize}
\item id - varchar(15)
\label{sec-2-2-1}
linked to Participants.badgeid

\item conid - int(11)
\label{sec-2-2-2}
linked to ConInfo.conid

\item created - datetime
\label{sec-2-2-3}
\begin{itemize}
\item when applied I think?
\item CORRECT
\end{itemize}

\item digital\_advertising - Y/N (enum)
\label{sec-2-2-4}
\begin{itemize}
\item y/n or \$ or type?
\item do they want it? or how much?
\item JUST A Y/N
\end{itemize}

\item fff\_sponsorship - Y/N (enum)
\label{sec-2-2-5}
\begin{itemize}
\item y/n or \$ or level?
\item will they sponsor us?
\item JUST A Y/N
\end{itemize}

\item print\_advertising - Y/N (enum)
\label{sec-2-2-6}
\begin{itemize}
\item y/n or \$ or type?
\item do they want it?
\item JUST A Y/N
\end{itemize}

\item updated - timestamp
\label{sec-2-2-7}
\begin{itemize}
\item most recently updated, I believe
\item CORRECT
\end{itemize}

\item vendor\_acknowledgement - varchar(25) (signature)
\label{sec-2-2-8}
\begin{itemize}
\item Vendor acknowledges \ldots{} something?
\item VENDOR ACKNOWLEDGES THEY AGREE TO OUR TERMS AND CONDITIONS -
THIS SHOULD BE AGREED TO ANNUALLY
\end{itemize}

\item vendor\_additional\_notes / additional\_information - text
\label{sec-2-2-9}
\begin{itemize}
\item this might also want to be perenial as well, perhaps?
\item ANNUAL ONLY; THIS IS REGARDING REQUESTS (EX:  FUR ALLERGY, WALL,
ETC)
\end{itemize}

\item vendor\_amenities\_foo - ?? (see also VendorFeatures)
\label{sec-2-2-10}
\begin{itemize}
\item A variety of possible fields?  How is this  used?
\begin{itemize}
\item 6ft\_table
\item corner\_endcap
\item extra\_badges
\item number\_of\_chairs
\item shared\_electrical 8ft\_tables
\end{itemize}
\item RIGHT NOW IT'S SPLIT INTO MULTIPLE LINES WITH MULTI OPTIONS.  \#
OF TABLES, \# OF CHAIRS, CORNER/END CAP, ELECTRICAL, EXTRA
BADGES
\end{itemize}

\item vendor\_contract - varchar(5)
\label{sec-2-2-11}
\begin{itemize}
\item How is this used?
\item THIS IS INITIALS TO STATE THEY READ AND AGREE TO OUR TERMS ALREADY ABOVE
\end{itemize}

\item vendor\_invoiced - Y/N (enum)
\label{sec-2-2-12}
\begin{itemize}
\item Were they invoced?
\item YUP
\end{itemize}

\item vendor\_location - ??
\label{sec-2-2-13}
\begin{itemize}
\item Where we put them.
\item YUP
\item This might want to be slightly more complex, building, room, and booth-number
\end{itemize}

\item vendor\_payment\_adjustment - decimal(13,4)
\label{sec-2-2-14}
\begin{itemize}
\item Adjustments to the bill
\item YUP
\end{itemize}

\item vendor\_payment\_amount - decimal(13,4)
\label{sec-2-2-15}
\begin{itemize}
\item amount they paid, should match invoiced amounts
\item YUP
\end{itemize}

\item vendor\_payment\_received - Y/N (enum)
\label{sec-2-2-16}
\begin{itemize}
\item check box?
\item YES
\item default "N" or maybe blank?
\end{itemize}

\item vendor\_preferred\_space - ?? (see also VendorSpaces)
\label{sec-2-2-17}
\begin{itemize}
\item Where they would like to be put.
\item NO - THIS IS BOOTH SIZE (SINGLE, DOUBLE, ETC)
\end{itemize}

\item vendor\_loadin\_self\_carry - Y/N (enum)
\label{sec-2-2-18}
\begin{itemize}
\item will they?
\item This should be Y/n, right?
\item YES
\end{itemize}

\item vendor\_sponsorship\_package  - ??
\label{sec-2-2-19}
\begin{itemize}
\item what package they wanted?  They actually bought?
\item THIS IS THE Y/N ABOVE
\item if this is redundant it should be removed.  If it is the same as
what is in missing, below, then \ldots{} perhaps not so missing after
all?
\end{itemize}

\item vendor\_status / status - Denied/Approved/Accepted/Approved/Paid (enum)
\label{sec-2-2-20}
\begin{itemize}
\item STATUS
\item DENIED, APPROVED, ACCEPTED, APPROVED AND PAID
\item Approved is in twice?  Should it be 4, Denied, Approved,
Accepted, Paid?  I'd also add "Proposed" or the like?
\end{itemize}

\item rejected\_reason - text
\label{sec-2-2-21}
\begin{itemize}
\item KEEP, WOULD BE SUBJECT TO CHANGE UNLESS VENDOR IS BANNER
\end{itemize}
\end{itemize}

\subsection{Missing}
\label{sec-2-3}
These are values that I think we need, that don't seem to actually
exist anywhere else, and should probably be at least tracked, if
not having several layers to them, so that they can be figured into
things like costs, and historical data.
\begin{itemize}
\item vendor\_sponsor\_level
\label{sec-2-3-1}
\begin{itemize}
\item This might be the vendor\_sponsorship\_package piece?
\item This might get more complex, depending on the offerings
\end{itemize}

\item digital\_advertising\_level
\label{sec-2-3-2}
\begin{itemize}
\item This might need to be a table of some sort, based on the
offerings
\end{itemize}

\item print\_advertising\_level
\label{sec-2-3-3}
\begin{itemize}
\item This might need to be a table of some sort, based on the
offerings
\end{itemize}

\item vendor\_actual\_space
\label{sec-2-3-4}
\begin{itemize}
\item single, double, room, etc. probably based on the same
VendorSpaces/VendorHasVendorSpace etc.
\end{itemize}

\item vendor\_location\_requests
\label{sec-2-3-5}
\begin{itemize}
\item although this might be under vendor notes?
\end{itemize}
\end{itemize}

\subsection{Generated}
\label{sec-2-4}
These values would not actually be stored anywhere, but generated
for various purposes, from other information collected in the
database, rather than simple values.  The Vendor Invoice Amount
should also go along with the Vendor Invoice, which lists all the
things they are paying for, and interfaces with our credit-card
system.
\begin{itemize}
\item vendor\_invoice\_amount - decimal(13,4)
\label{sec-2-4-1}
\begin{itemize}
\item Money
\item YUP
\item possibly a geneerated field, as opposed to a stored one, based
on the varieties chosen in vendor\_amenites,
vendor\_payment\_adjustment, vendor\_actual\_space and others?
\end{itemize}

\item times\_at\_fff
\label{sec-2-4-2}
\begin{itemize}
\item currently a freeform field, should probably be generated, with a
special number that adds in, for (currently) untracked events.
\item SOUNDS GOOD
\item Base figure above in perennial
\end{itemize}
\end{itemize}

\subsection{Structture}
\label{sec-2-5}
These are the proposed new tables or modifications to existing
tables that the data will be collected in.  Mostly interesting for
building purposes, but also for those curious on how things operate
under the hood, because, I like sharing that way, as opposed to
keeping things secret and cryptic.  Yes, I know it's called "code"
because we didn't want you to know what it did, but \ldots{} 
\begin{itemize}
\item Vendor Constants
\label{sec-2-5-1}
\begin{itemize}
\item Most of these (as listed) are already in the CongoDump table.
\item Two need their own table:
\begin{itemize}
\item vendor\_type
possibly a short list of possible with a mapping to the
vendor, so might be another table
\item times\_at\_fff\_untracked - int(3)
\end{itemize}
\end{itemize}

\item Vendor Yearly Variables
\label{sec-2-5-2}
\begin{itemize}
\item Most just are, as specified in this table
\item Some might be cross-references to the other tables below, or be
in the other tables entirely.
\end{itemize}

\item Vendor Features
\label{sec-2-5-3}
Probably just a tweak to this table, and might need to be a
"standard set" and a "this year set" like BaseFeatures and
Features does as well as BaseServices and Services does and a
repurposing of "SessionHasVendorFeature" to
"VendorHasVendorFeature"

\item Vendor Spaces
\label{sec-2-5-4}
Probably just a tweak to this table, and might need to be a
"standard set" and a "this year set" like BaseFeatures and
Features does as well as BaseServices and Services does and a
repurposing of "SessionHasVendorSpace" to "VendorHasVendorSpace"

\item Vendor Location
\label{sec-2-5-5}
Might be it's own table, with conid (linked), badgeid (linked),
building, floor, room (possibly linked), and booth\_number, and
then code around it so blanks just don't get reported (aka, if
they are a room vendor, their booth\_number might not be listed,
but if there are several vendors in a room, it might be.)  This
might also allow ties to maps, and possibly the redraw of maps.

\item Vendor Status
\label{sec-2-5-6}
Might want to be it's own table as a mapping, so searches, and
display\_order could be set.

\item Vendor Sponsorship Package/vendor\_sponsor\_level
\label{sec-2-5-7}
These three would need to be determined.

\item digital\_advertising\_level
\label{sec-2-5-8}
These three would need to be determined.

\item print\_advertising\_level
\label{sec-2-5-9}
These three would need to be determined.
\end{itemize}

\subsection{Still Unknown}
\label{sec-2-6}
These are the variables that exist already in the other usage
space, that I still don't have a handle on.  More explination would
be very useful.
\begin{itemize}
\item created\_by - number
\label{sec-2-6-1}
\begin{itemize}
\item Not sure what this does?
\item I THINK IT"S VENDORS NAME VS. I UPDATED, ETC
\item So \ldots{} how should it be expressed/tracked?
\end{itemize}

\item ordering\_count - number
\label{sec-2-6-2}
\begin{itemize}
\item Another random number?
\item YES
\item Should it be kept?  Removed, what is it's state?  What is it's
purpose?
\end{itemize}
\end{itemize}

\section{Front End}
\label{sec-3}
\subsection{Available Pages}
\label{sec-3-1}
This should be the comprehensive list of all the pages that either
the vendors/community table folk need to interact with or those who
manage them (the Div Head) needs to interact with.
\begin{itemize}
\item Returning Vendor sign-in
\label{sec-3-1-1}
By number or email address.

\item Returning Community Table sign-in
\label{sec-3-1-2}
By number or email address.  Might simply be switched on one of
the values in the vending type or on it's own value, and just have
one place to go?  Since we are collecting pretty much the same
information for them both?  Or should I be separating them out?

\item New Vendor proposal
\label{sec-3-1-3}
With a catch to see if they are already in our database, using
their email address

\item New Community Table proposal
\label{sec-3-1-4}
With a catch to see if they are already in our database, using
their email address

\item Returning Vendor/Community Table informational gathering
\label{sec-3-1-5}
This can be the same, although there might, again, be a slight
difference bases on vendor/community table differences.  Also, we
might want to present last? event's information as defaults, so
they have to do less typing. Dealing with changes once the state
has changed to "invoice generated" might require Vendor Div Head
intervention.

\item Returning Vendor informational update
\label{sec-3-1-6}
This might not be necessary, based on the above page keeping what
is set for this year as set.

\item Returning Vendor/Community Table Bio update
\label{sec-3-1-7}
Page that they can, at will? update their web or book information.

\item Invoice
\label{sec-3-1-8}
Not available until after both they are accepted (by Vendor
Div Head), and they hit the "generate invoice" button, which
probably should lock changes.  Once the invoice is generated, it
should be fixed, and also payable.
This will (of course) be a comprehensive list of everything they
agreed to.  The invoice page should continue to be available after
they paid, so there is a referece point to the services agreed
to/paid for.

\item Ad Copy Submissions
\label{sec-3-1-9}
For those who have agreed to either digital or print ads, this is
a place for them to upload/submit them.

\item Vendor Jury
\label{sec-3-1-10}
The page that all the proposed vendors, with certain bits of
information (to be determined, since I've not seen that page yet
in our current system) will be on for the Vendor Div Head to move
vendors to either "Accepted" or "Denied".  Probably includes some
vendor notes and the like, including "Denied Reasons" Also
probably where any price adjustments happen to the invoice.  Maybe
a grouping by Vendor Type, and leaving the ones already moved on
from this in a different colour, so that those already accepted,
and their type, will be shown for balance reasons.

\item Vendor Placement
\label{sec-3-1-11}
After the invoice is paid? Or perhaps slightly different, if there
is those that have invoice exceptions allowing for the placement
of vendors.  Probably should have the invoice amount paid, and
invoice amount expected on this page, to verify that they are,
indeed, all paid up.  Also the vendor type should show up here, as
well, so that similars can either be grouped together, or spread
apart, as per the decision of the Div Head. May also contain
previous placements, or links to previous maps or something useful
along those lines.

\item Administer Vendor
\label{sec-3-1-12}
Where any changes are made to the vendor information (both that
they can change and that they can't) for creation/update by Vendor
Div Head.

\item Vendor Welcome Letter
\label{sec-3-1-13}
Generate the (physical) welcome letter to the vendors

\item Vendor Email
\label{sec-3-1-14}
Generate (and send) various all-vendor communications

\item Vendor Badges
\label{sec-3-1-15}
Generate the Vendor Badges for printing purposes.

\item Vendor Tents
\label{sec-3-1-16}
Generate all the vendor "tents" to put on their tables, to mark
their place.

\item Check-in/Check-out lists
\label{sec-3-1-17}
All of the vendors expected, so they can be ticked off the list
when they get there, and, so that when they return whatever they
have to return at the end of the night (tax forms, possibly?) can
be ticked off again.
\end{itemize}

\subsection{Flow}
\label{sec-3-2}
\begin{itemize}
\item Task List
\label{sec-3-2-1}
Task List element with the call for vendors is created, with the
appropriate link.

\item Task List
\label{sec-3-2-2}
Task List element with the call for community tables is created,
with the appropriate link.

\item Email
\label{sec-3-2-3}
Possible email from Vendor Div Head goes out reminding
vendors/community tables when the call will open and close.

\item Phase Change
\label{sec-3-2-4}
At the appropriate time, the Con Chair or Zambia bitch open the
call for vendors.

\item Email
\label{sec-3-2-5}
Vendor Div Head generates and sends email letting the
vendors/community tables know that the call is open.

\item Vendors/Community Table folk apply
\label{sec-3-2-6}
They use either the returning or the new pages, select the
appropriate features, sponsorships, and ads that they want, and
make sure everything is as expected.

\item (If Rolling Acceptance) Jury
\label{sec-3-2-7}
If there is a rolling acceptantce of vendors and/or community
tables, while the vendor call/community table call is open
acceptances can roll out and then things can go forward.
Otherwise \ldots{}

\item Phase Change
\label{sec-3-2-8}
At the appropriate time, the Con Chair or Zambia bitch close the
call for vendors.

\item Jury
\label{sec-3-2-9}
The (rest of (if rolling acceptance is happening)) vendors and
community tables get acceptance, which probably is an individual
email from the Div Head, with any specific to them in information
included, and an invitation for them to pay their invoice.

\item Invoice
\label{sec-3-2-10}
Vendors then (hopefully, magically, smoothly) pay their agreed-to
invoices.

\item Ad Copy
\label{sec-3-2-11}
All paid for advertizing copy is generated by the vendors or
community tables is submitted.

\item Placement
\label{sec-3-2-12}
Div Head (and anyone else that can be roped into helping)
schedules the vendors to be put in the appropriate locations on
the day of the event.

\item Email
\label{sec-3-2-13}
Final email goes out, with any instruction reminders and if there
is a manditory vendor meeting, or the like.

\item Welcome Letter
\label{sec-3-2-14}
Welcome lettters generated and printed, possibly stuffed in an
envelope or folder of some sort.
This might include some form of their (electronically) signed
contract, and maybe invoice highlights.  Definitely includes their
placement information, and information on how load-in and load-out
works, etc.

\item Badges
\label{sec-3-2-15}
Badges, the appropriate number for each vendor and community
table, are printed, and possibly stuffed as above.

\item Wrist bands
\label{sec-3-2-16}
Are possibly stuffed as above.

\item Check-in/check-out lists printed
\label{sec-3-2-17}
So they are available on the day-of.
\end{itemize}

\section{Conclusion}
\label{sec-4}
This should cover all the bits and pieces.  If I missed any, let me
know so this document can be updated, and the system can be tested.
Hopfully all will work, and this can go forth as The New Way.  Until
we change it, again, in 2-3 years \textbf{grin}.
\end{document}